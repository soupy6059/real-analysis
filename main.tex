\documentclass[12pt]{article}
\usepackage{my_preamble}

% Document Starts
\begin{document}

\maketitle 

\begin{abstract}
  Real Analysis the study of approximation on the reals.
\end{abstract}

\tableofcontents

\pagebreak
% <++>

\section{Cardinality}
\subsection{Brief Motivation}
We want to build a metric space to measure the distance between objects.

We need
\begin{enumerate}
  \item set $X$ of objects.
  \item need to measure closeness. func \(d:X\times X\to[0,\infty)\) s/t
    \begin{enumerate}
      \item \(d(x,y)=0\iff x=y\)
      \item \(d(x,y)=d(y,x)\)
      \item \(d(x,z)\le d(x,y)+d(y,z)\)
    \end{enumerate}
\end{enumerate}

We call \(d\) a metric on \(X\). \((X,d)\) is a \TB{metric space}.
\bboxex
\((\R,+'\circ^\circ(\sqrt{\,}^=)-)\) is a metric space. Note the BQN notation.
\ebox

\subsection{Function Theory}
\bbox
\begin{defn}[Injection] \label{defn:inj}
  Let \(A,B\) be non-empty sets. We say \(f:A\to B\) is injective \TB{iff}
  \(\forall a,b\in A\quad f(a)=f(b)\implies a=b\)
\end{defn}
\ebox

\bbox
\begin{defn}[Surjection] \label{defn:surj}
  \(f:A\to B\) is a surjection if \(\forall b\in B\quad\exists a\in A\st f(a)
  = b\).
\end{defn}
\ebox

\bbox
\begin{defn}[Bijective] \label{defn:bij}
  \(f:A\to B\) is bijective \TB{iff} its injective and surjective.
\end{defn}
\ebox


\bbox
\begin{defn}[Invertable]
  \(f:A\to B\) is invertable \TB{iff} \(\exists g:B\to A\st g(f(a))=a\T{ and }
  f(g(b))=b\quad\forall a\in A,\,b\in B\).
\end{defn}
\ebox
We write \(g=f^{-1}\) and call it "the" inverse.


\bbox
\begin{prop}
  \(f:A\to B\) is invertable \TB{iff} f is bijective.
\end{prop}
\ebox

\begin{proof}
  \((\implies)\) \(f\) is invertable. Suppose \(f(a)=f(b)\). We'll show \(a=b\).
  \[f^{-1}f(a))=f^{-1}f(b))\]
  \[\implies a=b\]
  
  Now we'll show \(\forall b\in B\;\exists a\in A\;f(a)=b\).
  \[a=f^{-1}(b)\implies\T{there is way to get from b to a, and it's }f^{-1}\]

  \((\limplies)\) Assume \(f\is(\T{bijective})\). We'll construct \(f\)'s
  inverse. For \(b\in B\) let \(a_b\) be the unique element of \(A\st f(a_b)=b\).
  \(a_b\) exists b/c of surjectivity of f, and it's unique b/c of injectivity.
  \[g:=\{g:A\to B,\,g(b)=a_b\}\]
  \[f(g(b))=f(a_b)=b\]
  \[g(f(a_b))=g(b)=a_b\]
  \[\implies g=f^{-1}\]
\end{proof}

\bbox
\begin{prop}
  \(\exists(\T{injection})\,f:A\to B\iff\exists(\T{surjection})\,g:B\to A\)
\end{prop}
\ebox

\bboxproof
\begin{proof}
  \((\implies)\) Suppose \(f:A\to B\is(\T{injective})\). Let \(b\in B\).
  
  Case 1: \(b\in f(A)\). 

  Let \(g(b)\) be the unique element of \(A\st f(g(b))=b\), unique b/c 
  \(f\is\TM{injective}\)

  Case 2: \(b\not\in f(A)\).

  Fix any \(z\in A\). Let \(g(b)=z\).
  
  \begin{equation*}
    \implies g(b)=\begin{cases}f^{=}(b)&b\in f(A)\\ z&b\not\in f(A)\end{cases}
  \end{equation*}

  We claim \(g\) is a surjection. 
  So we have to show
  \(\forall a\in A,\,\exists b\in B\st g(b)=a\)
  Let \(a\in A\st f(a)\in B\).
  \[g(f(a))\implies f(g(f(a)))=f(a)\]
  \[\TM{injective}\implies g(f(a))=a\]
  \[\implies g\is\TM{surjective}\]

  \(\limplies\) Suppose \((g:B\to A)\is\TM{surjective}\).
  \(\forall a\in A\T{ choose }b_a\in B\st g(b_a)=a\).
  \(f:=\{f:A\to B\quad f(a)=b_a\}\). Suppose
  \[f(x)=f(y)\]
  \[\implies b_x=b_y\]
  \[\implies g(b_x)=g(b_y)\]
  \[\implies x=y\]
  \[\implies f\is\TM{injective}\]
\end{proof}
\ebox

% lecture 2


\bbox
\begin{defn}[Powerset]
  Let \(X\) be a set. Then \(\PP(X):=\{A:A\subseteq X\}\), called
  the "\TB{powerset} of \(X\)."
\end{defn}
\ebox

\bboxex
\[X=\{a,b\}\]
\[\PP(X)=\{\emptyset,\{a\},\{b\},\{a,b\}\}\]
\ebox


\bbox
\begin{axim}[Choice]\label{axim:choice}
  Given \(X\neq\emptyset\exists\T{a choice func }f:\PP(X)\backslash
  \{\emptyset\}\to X\st f(A)\in A\;\forall\empty\neq A\subseteq A\).
\end{axim}
\ebox

\subsection{Cardinality}
\bboxex
\[A=\{a,b\},\;B=\{c,d,e,f\}\]
\[\T{Intuitively }|A|<|B|\]
\[f\is\TM{inj}:A\to B,\;f(a):=c,\;f(b):=d\]
\[\implies f\is\TM{inj}(A)\subset B\]
\[\implies |A|\le |B|\]
\ebox

\bbox
\begin{defn}[Ordering of Cardinality on Sets]
  \(A,B\) sets.
  \begin{enumerate}
    \item \(|A|\le|B|\iff\exists f\is\TM{inj}:A\to B\)
    \item \(|A|=|B|\iff\exists f\is\TM{bij}:A\to B\)
  \end{enumerate}
\end{defn}
\ebox
\bboxex
\[|\N|\le|\Z|\limplies f\is\TM{inj}:\N\to\Z,\;f(n):=n\]
\begin{equation*}
  f\is\TM{bij}:\N\to\Z:f(n):=
  \begin{cases}
    2n+2&:n\ge0\\
    2(-n)-1&:n<0
  \end{cases}\implies|\N|=|\Z|
\end{equation*}
\[h\is\TM{bij}:\R\to(0,1):h(x):=\frac{\mathrm{arctan}(x)+\pi/2}\pi\implies|\R|=|(0,1)|\]
\ebox

\bbox
\begin{thm}[Cantor-Schroeder-Berstien (CSB)] \label{thm:CSB}
  if \(|A|\le|B|\) and \(|B|\le|A|\) then \(|A|=|B|\).
\end{thm}
\ebox


\bbox
\begin{lem}[Phi has a Fixed Point]\label{lem:phi_fixed_pt}
  \(X\) set. Suppose \(\exists\phi:\PP(X)\to\PP(X)\st\phi(A)\subseteq\phi(B)
  \T{ if }A\subseteq B\subseteq X\). Then
  \[\exists F\subseteq X\st\phi(F)=F\]
\end{lem}
\ebox

\bboxproof
Let \(F=\bigcup_{A\subseteq X:A\subseteq\phi(A)}A\).
\bboxnote
\TB{Note:} \(\emptyset\subseteq X\;\&\;\emptyset\subseteq\phi(\emptyset)\)
\ebox
\TB{Claim:} \(F=\phi(F)\).
Take \(A\subseteq X\) with \(A\subseteq\phi(A)\). Then \(A\subseteq F\).
\[\implies\phi(A)\subseteq\phi(F)\]
\[\implies A\subseteq\phi(F)\]
\[\implies\bigcup_{A\subseteq X:A\subseteq\phi(A)}A\subseteq\phi(F)\quad
\T{(by properties of unions)}\]
\[\implies F\subseteq\phi(F)\]
Further,
\[F\subseteq \phi(F)\implies\phi(F)\subseteq\phi(\phi(F))\]
\[\implies\phi(F)\in\{A\subseteq X:A\subseteq\phi(A)\}\]
For this step, let \(\phi(F)=B\) and notice that \(B\subseteq\phi(B)\). So
its in the set above.
\[\implies\phi(F)\subseteq\bigcup_{A\subseteq X:A\subseteq\phi(A)}A=F\]

\[\implies F=\phi(F)\]
\ebox

\bboxex
\TB{To motivate (CSB)~\cref{thm:CSB}:} prove that \(|N|=|N\times N|\).
\[f\is\TM{inj}:\N\to\N\times\N,\;f(n):=(n,1)\]
\[g\is\TM{inj}:\N\times\N\to\N,\;g((n,m)):=2^n3^m\]
By (CSB) \(|\N|=|\N\times\N|\).
\ebox

% lecture 3

\bboxproof
\begin{proof} of (CSB)~\cref{thm:CSB}.
  Let \(f,g\is\TM{inj}:A:B\). For \(X\subseteq Y\subseteq A\)
\[f(X)\subseteq f(Y)\]
\[\implies B\backslash f(Y)\subseteq B\backslash f(X)\]
\[\implies g(B\backslash f(Y))\subseteq g(B\backslash f(X))\]
\[\implies A\backslash g(B\backslash f(X))\subseteq A\backslash g(B\backslash f(Y))\]
This letting \(\phi:\PP(A)\to\PP(A):\phi(x):=A\backslash g(B\backslash f(x))\)
insures it preserves \(\subseteq\).
So by the \cref{lem:phi_fixed_pt}, \(\exists F\subseteq A
\st F=\phi(F)=A\backslash g(B\backslash f(F))\). In particular, \(A
\backslash F=g(B\backslash f(F))\implies g:B\backslash f(F)\to A\backslash F
:\is\TM{bij}\).
\bboxnote
\TB{Note.} It's a surjection b/c everyone in \(A\backslash F\) gets mapped to
b/c it's the image if \(g(B\backslash f(F))\).
\ebox
Moreover, \(\inv g:A\backslash F\to B\backslash f(F)\) is a bijection, and
\(f:F\to f(F)\) is a bijection (for the same reason as above; restriction of
domain of an injective function is injective, and a function that maps to its image is
automatically a surjection). Hence
\begin{equation*}
  h:A\to B:h(x):=
  \begin{cases}
    \inv g(x) &:x\in A\backslash F\\
    f(x) &:x\in F
  \end{cases}
\end{equation*}
\end{proof}
\ebox

\bboxex
Show \(|\Q|=|\N|\).
\bboxproof
\begin{proof}
  \[f:\N\to\Q:f(x):=x\implies|\N|\le|\Q|\]
  \(q\in\Q\) can be written in the form \(q=\frac m n,\;m\in\Z,\,n\in\N\).
  \[g:\Q\to\Z\times\N:g(q):=(m,n)\;:\;q=\frac m n\]
  \[\implies|\Q|\le|\Z\times\N|=|\N\times\N|=|\N|\]
  So by (CSB) \cref{thm:CSB}, \(|\Q|=|\N|\).
\end{proof}
\ebox
\ebox


\bbox
\begin{defn}[Finite, Countably Infinite, Countable]
  \(\,\)\begin{enumerate}
    \item a set \(A\) is \TB{finite} \TU{iff} \(|A|=|\{1,2,\dots,n\}|\)
      for some \(n\in\N\). In this case, \(|A|=n\).
    \item \(|\emptyset|:=0\)
    \item \(A\) is \TB{countably infinite} \TU{iff} 
      \(|A|=|\N|:=\aleph_0\).
    \item \(A\) is countable \TU{iff} \(A\) is finite or ctbly infinite.
  \end{enumerate}
\end{defn}
\ebox

\bboxex
\[|\N|=|\Z|=|\N\times\N|=|\Q|=\aleph_0\]
\ebox


\bbox
\begin{prop}[Aleph Null is the Smallest Infinity]\label{prop:aleph_null_small}
  If \(A\) is infinite, then \(|\N|\le|A|\).
\end{prop}
\ebox

\bboxproof
\begin{proof}
  By (Choice) \cref{axim:choice}, \(\exists f:\PP(A)\backslash\{\emptyset\}\to
  A\st f(X)\in X,\;\forall\emptyset\neq X\subseteq A\).
  \begin{equation*}
    \begin{split}
      \T{Let }a_1&=f(A)\in A\\
      a_2&=f(A\backslash\{a_1\})\in A\backslash\{a_1\}\\
         &\vdots
    \end{split}
  \end{equation*}
  \(\implies\aleph_0=|\{a_1,\dots\}|\le|A|\).
\end{proof}
\ebox

\bboxex
\(\R\) is uncountable. So \(\not\exists f\is\TM{bij}:\N\to\R\).
\bboxproof
  \begin{proof}
    Since \(|\R|=|(0,1)|\), we'll show that \((0,1)\) is uncountable.
  \end{proof}
\ebox
\ebox


% <++>
\end{document}


























% scrolloff buffer
