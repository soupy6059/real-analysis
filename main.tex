\documentclass[12pt]{article}
\usepackage{my_preamble}

% Document Starts
\begin{document}

\maketitle 

\begin{abstract}
  Real Analysis the study of approximation on the reals.
\end{abstract}

\tableofcontents

\pagebreak
% <++>

\section{Cardinality}
\subsection{Brief Motivation}
We want to build a metric space to measure the distance between objects.

We need
\begin{enumerate}
  \item set $X$ of objects.
  \item need to measure closeness. func \(d:X\times X\to[0,\infty)\) s/t
    \begin{enumerate}
      \item \(d(x,y)=0\iff x=y\)
      \item \(d(x,y)=d(y,x)\)
      \item \(d(x,z)\le d(x,y)+d(y,z)\)
    \end{enumerate}
\end{enumerate}

We call \(d\) a metric on \(X\). \((X,d)\) is a \TB{metric space}.
\bboxex
\((\R,+'\circ^\circ(\sqrt{\,}^=)-)\) is a metric space. Note the BQN notation.
\ebox

\subsection{Function Theory}
\bbox
\begin{defn}[Injection] \label{defn:inj}
  Let \(A,B\) be non-empty sets. We say \(f:A\to B\) is injective \TB{iff}
  \(\forall a,b\in A\quad f(a)=f(b)\implies a=b\)
\end{defn}
\ebox

\bbox
\begin{defn}[Surjection] \label{defn:surj}
  \(f:A\to B\) is a surjection if \(\forall b\in B\quad\exists a\in A\st f(a)
  = b\).
\end{defn}
\ebox

\bbox
\begin{defn}[Bijective] \label{defn:bij}
  \(f:A\to B\) is bijective \TB{iff} its injective and surjective.
\end{defn}
\ebox


\bbox
\begin{defn}[Invertable]
  \(f:A\to B\) is invertable \TB{iff} \(\exists g:B\to A\st g(f(a))=a\T{ and }
  f(g(b))=b\quad\forall a\in A,\,b\in B\).
\end{defn}
\ebox
We write \(g=f^{-1}\) and call it "the" inverse.


\bbox
\begin{prop}
  \(f:A\to B\) is invertable \TB{iff} f is bijective.
\end{prop}
\ebox

\begin{proof}
  \((\implies)\) \(f\) is invertable. Suppose \(f(a)=f(b)\). We'll show \(a=b\).
  \[f^{-1}f(a))=f^{-1}f(b))\]
  \[\implies a=b\]
  
  Now we'll show \(\forall b\in B\;\exists a\in A\;f(a)=b\).
  \[a=f^{-1}(b)\implies\T{there is way to get from b to a, and it's }f^{-1}\]

  \((\limplies)\) Assume \(f\is(\T{bijective})\). We'll construct \(f\)'s
  inverse. For \(b\in B\) let \(a_b\) be the unique element of \(A\st f(a_b)=b\).
  \(a_b\) exists b/c of surjectivity of f, and it's unique b/c of injectivity.
  \[g:=\{g:A\to B,\,g(b)=a_b\}\]
  \[f(g(b))=f(a_b)=b\]
  \[g(f(a_b))=g(b)=a_b\]
  \[\implies g=f^{-1}\]
\end{proof}

\bbox
\begin{prop}
  \(\exists(\T{injection})\,f:A\to B\iff\exists(\T{surjection})\,g:B\to A\)
\end{prop}
\ebox

\bboxproof
\begin{proof}
  \((\implies)\) Suppose \(f:A\to B\is(\T{injective})\). Let \(b\in B\).
  
  Case 1: \(b\in f(A)\). 

  Let \(g(b)\) be the unique element of \(A\st f(g(b))=b\), unique b/c 
  \(f\is\TM{injective}\)

  Case 2: \(b\not\in f(A)\).

  Fix any \(z\in A\). Let \(g(b)=z\).
  
  \begin{equation*}
    \implies g(b)=\begin{cases}f^{=}(b)&b\in f(A)\\ z&b\not\in f(A)\end{cases}
  \end{equation*}

  We claim \(g\) is a surjection. 
  So we have to show
  \(\forall a\in A,\,\exists b\in B\st g(b)=a\)
  Let \(a\in A\st f(a)\in B\).
  \[g(f(a))\implies f(g(f(a)))=f(a)\]
  \[\TM{injective}\implies g(f(a))=a\]
  \[\implies g\is\TM{surjective}\]

  \(\limplies\) Suppose \((g:B\to A)\is\TM{surjective}\).
  \(\forall a\in A\T{ choose }b_a\in B\st g(b_a)=a\).
  \(f:=\{f:A\to B\quad f(a)=b_a\}\). Suppose
  \[f(x)=f(y)\]
  \[\implies b_x=b_y\]
  \[\implies g(b_x)=g(b_y)\]
  \[\implies x=y\]
  \[\implies f\is\TM{injective}\]
\end{proof}
\ebox

% lecture 2


\bbox
\begin{defn}[Powerset]
  Let \(X\) be a set. Then \(\PP(X):=\{A:A\subseteq X\}\), called
  the "\TB{powerset} of \(X\)."
\end{defn}
\ebox

\bboxex
\[X=\{a,b\}\]
\[\PP(X)=\{\emptyset,\{a\},\{b\},\{a,b\}\}\]
\ebox


\bbox
\begin{axim}[Choice]\label{axim:choice}
  Given \(X\neq\emptyset\exists\T{a choice func }f:\PP(X)\backslash
  \{\emptyset\}\to X\st f(A)\in A\;\forall\empty\neq A\subseteq A\).
\end{axim}
\ebox

\subsection{Cardinality}
\bboxex
\[A=\{a,b\},\;B=\{c,d,e,f\}\]
\[\T{Intuitively }|A|<|B|\]
\[f\is\TM{inj}:A\to B,\;f(a):=c,\;f(b):=d\]
\[\implies f\is\TM{inj}(A)\subset B\]
\[\implies |A|\le |B|\]
\ebox

\bbox
\begin{defn}[Ordering of Cardinality on Sets]
  \(A,B\) sets.
  \begin{enumerate}
    \item \(|A|\le|B|\iff\exists f\is\TM{inj}:A\to B\)
    \item \(|A|=|B|\iff\exists f\is\TM{bij}:A\to B\)
  \end{enumerate}
\end{defn}
\ebox
\bboxex
\[|\N|\le|\Z|\limplies f\is\TM{inj}:\N\to\Z,\;f(n):=n\]
\begin{equation*}
  f\is\TM{bij}:\N\to\Z:f(n):=
  \begin{cases}
    2n+2&:n\ge0\\
    2(-n)-1&:n<0
  \end{cases}\implies|\N|=|\Z|
\end{equation*}
\[h\is\TM{bij}:\R\to(0,1):h(x):=\frac{\mathrm{arctan}(x)+\pi/2}\pi\implies|\R|=|(0,1)|\]
\ebox

\bbox
\begin{thm}[Cantor-Schroeder-Berstien (CSB)] \label{thm:CSB}
  if \(|A|\le|B|\) and \(|B|\le|A|\) then \(|A|=|B|\).
\end{thm}
\ebox


\bbox
\begin{lem}[Phi has a Fixed Point]\label{lem:phi_fixed_pt}
  \(X\) set. Suppose \(\exists\phi:\PP(X)\to\PP(X)\st\phi(A)\subseteq\phi(B)
  \T{ if }A\subseteq B\subseteq X\). Then
  \[\exists F\subseteq X\st\phi(F)=F\]
\end{lem}
\ebox

\bboxproof
Let \(F=\bigcup_{A\subseteq X:A\subseteq\phi(A)}A\).
\bboxnote
\TB{Note:} \(\emptyset\subseteq X\;\&\;\emptyset\subseteq\phi(\emptyset)\)
\ebox
\TB{Claim:} \(F=\phi(F)\).
Take \(A\subseteq X\) with \(A\subseteq\phi(A)\). Then \(A\subseteq F\).
\[\implies\phi(A)\subseteq\phi(F)\]
\[\implies A\subseteq\phi(F)\]
\[\implies\bigcup_{A\subseteq X:A\subseteq\phi(A)}A\subseteq\phi(F)\quad
\T{(by properties of unions)}\]
\[\implies F\subseteq\phi(F)\]
Further,
\[F\subseteq \phi(F)\implies\phi(F)\subseteq\phi(\phi(F))\]
\[\implies\phi(F)\in\{A\subseteq X:A\subseteq\phi(A)\}\]
For this step, let \(\phi(F)=B\) and notice that \(B\subseteq\phi(B)\). So
its in the set above.
\[\implies\phi(F)\subseteq\bigcup_{A\subseteq X:A\subseteq\phi(A)}A=F\]

\[\implies F=\phi(F)\]
\ebox

\bboxex
\TB{To motivate (CSB)~\cref{thm:CSB}:} prove that \(|N|=|N\times N|\).
\[f\is\TM{inj}:\N\to\N\times\N,\;f(n):=(n,1)\]
\[g\is\TM{inj}:\N\times\N\to\N,\;g((n,m)):=2^n3^m\]
By (CSB) \(|\N|=|\N\times\N|\).
\ebox

% lecture 3

\bboxproof
\begin{proof}[Proof of (CSB)~\cref{thm:CSB}]
  Let \(f,g\is\TM{inj}:A:B\). For \(X\subseteq Y\subseteq A\)
\[f(X)\subseteq f(Y)\]
\[\implies B\backslash f(Y)\subseteq B\backslash f(X)\]
\[\implies g(B\backslash f(Y))\subseteq g(B\backslash f(X))\]
\[\implies A\backslash g(B\backslash f(X))\subseteq A\backslash g(B\backslash f(Y))\]
This letting \(\phi:\PP(A)\to\PP(A):\phi(x):=A\backslash g(B\backslash f(x))\)
insures it preserves \(\subseteq\).
So by the \cref{lem:phi_fixed_pt}, \(\exists F\subseteq A
\st F=\phi(F)=A\backslash g(B\backslash f(F))\). In particular, \(A
\backslash F=g(B\backslash f(F))\implies g:B\backslash f(F)\to A\backslash F
:\is\TM{bij}\).
\bboxnote
\TB{Note.} It's a surjection b/c everyone in \(A\backslash F\) gets mapped to
b/c it's the image if \(g(B\backslash f(F))\).
\ebox
Moreover, \(\inv g:A\backslash F\to B\backslash f(F)\) is a bijection, and
\(f:F\to f(F)\) is a bijection (for the same reason as above; restriction of
domain of an injective function is injective, and a function that maps to its image is
automatically a surjection). Hence
\begin{equation*}
  h:A\to B:h(x):=
  \begin{cases}
    \inv g(x) &:x\in A\backslash F\\
    f(x) &:x\in F
  \end{cases}
\end{equation*}
\end{proof}
\ebox

\bboxex
Show \(|\Q|=|\N|\).
\bboxproof
\begin{proof}
  \[f:\N\to\Q:f(x):=x\implies|\N|\le|\Q|\]
  \(q\in\Q\) can be written in the form \(q=\frac m n,\;m\in\Z,\,n\in\N\).
  \[g:\Q\to\Z\times\N:g(q):=(m,n)\;:\;q=\frac m n\]
  \[\implies|\Q|\le|\Z\times\N|=|\N\times\N|=|\N|\]
  So by (CSB) \cref{thm:CSB}, \(|\Q|=|\N|\).
\end{proof}
\ebox
\ebox


\bbox
\begin{defn}[Finite, Countably Infinite, Countable]
  \(\,\)\begin{enumerate}
    \item a set \(A\) is \TB{finite} \TU{iff} \(|A|=|\{1,2,\dots,n\}|\)
      for some \(n\in\N\). In this case, \(|A|=n\).
    \item \(|\emptyset|:=0\)
    \item \(A\) is \TB{countably infinite} \TU{iff} 
      \(|A|=|\N|:=\aleph_0\).
    \item \(A\) is countable \TU{iff} \(A\) is finite or ctbly infinite.
  \end{enumerate}
\end{defn}
\ebox

\bboxex
\[|\N|=|\Z|=|\N\times\N|=|\Q|=\aleph_0\]
\ebox


\bbox
\begin{prop}[Aleph Null is the Smallest Infinity]\label{prop:aleph_null_small}
  If \(A\) is infinite, then \(|\N|\le|A|\).
\end{prop}
\ebox

\bboxproof
\begin{proof}
  By (Choice) \cref{axim:choice}, \(\exists f:\PP(A)\backslash\{\emptyset\}\to
  A\st f(X)\in X,\;\forall\emptyset\neq X\subseteq A\).
  \begin{equation*}
    \begin{split}
      \T{Let }a_1&=f(A)\in A\\
      a_2&=f(A\backslash\{a_1\})\in A\backslash\{a_1\}\\
         &\vdots
    \end{split}
  \end{equation*}
  \(\implies\aleph_0=|\{a_1,\dots\}|\le|A|\).
\end{proof}
\ebox

\bbox
\begin{prop}[The Reals are Uncountable]\label{prop:uncountable_reals}
\(\R\) is uncountable. So \(\not\exists f\is\TM{bij}:\N\to\R\).
\bboxproof
  \begin{proof} \TB{Cantor's Diagonal Element Proof.} 
    Since \(|\R|=|(0,1)|\), we'll show that \((0,1)\) is uncountable.

    For contradiction, assume that there exists a bijection \(f:\N\to(0,1)\).

    So let's say
    \begin{equation*}
      \begin{split}
        f(1)&=0.a_{11}a_{12}a_{13}\cdots\\
        f(2)&=0.a_{21}a_{22}a_{23}\cdots\\
        f(3)&=0.a_{31}a_{32}a_{33}\cdots\\
        \vdots&=\vdots
      \end{split}
    \end{equation*}
    Where we avoid repeated nines.

    Choose \(b_i\in\{1,\dots,8\}\st b_i\neq a_{ii}\).
    \[\implies\not\exists n\in\N,\;f(n)=0.b_1b_2b_3\cdots\]
    Thats a contradiction.
  \end{proof}
\ebox
\end{prop}
\ebox


\bbox
\begin{defn}[Continuum]\label{defn:continuum}
  We write \(|\R|=c\), where \(c\) stands for \TB{continuum}.
\end{defn}
\ebox

So we have 3 cardinals: \(n,\aleph_0,c\).


\bbox
\begin{axim}[Continuum Hypothesis]\label{axim:continuum_hypth}
  If \(A\) is a set with \(\aleph_0\le|A|\le c\), then
  \(\aleph_0=|A|\) or \(|A|=c\).
\end{axim}
\ebox

\subsection{Cardinality of Power Sets}


\bbox
\begin{prop}
  If \(|A|=n\), then \(|\PP(A)|=2^n\).
\end{prop}
\ebox
\bboxproof
\begin{proof}[Proof]
  \[|\PP(A)|=\sum_{k=1}^n\binom nk=(1+1)^n=2^n\]
\end{proof}
\ebox


% lecture 4



\bbox
\begin{defn}[Cartesian Product]\label{defn:cartesian_product}
  Let \(I\) be a set. \(\forall i \in I\;\T{Let }A_i\is\TM{set}\implies
  \prod_{i\in I}A_i:=\{f|f:I\to\bigcup A_i,\;f(i)\in A_i\}\)
\end{defn}
\ebox


\bboxnote
\[f(i)\in A_i\]
\[I=\N\implies f:\N\to\bigcup A_i:f(i):\!\in A_i\equiv(f(1),f(2),\dots)\]
\ebox

\bbox
\begin{defn}[Set Power]\label{defn:set_power}
  \(A,B\is\TM{set}\implies A^B=\{f:B\to A\}\)
  \[|A|^{|B|}:=\left|A^B\right|=|\{f:B\to A\}|\]
\end{defn}
\ebox

\bbox
\begin{prop}[Cardinality of a Power Set]\label{prop:card_of_powerset}
  if \(X\is\TM{set}\), \(\PP(X)=2^{|X|}=|\{f:X\to\{0,1\}\}|\).
\end{prop}
\ebox

\bboxproof
\begin{proof}
  \begin{equation*}
    \begin{split}
      \phi:\PP(X)\to\{f:X\to\{0,1\}\}:\phi(A)&:=\chi_A\\
      \chi_A:X\to\{0,1\}:\chi_A(x)&:=
      \begin{cases}
        1 &: x\in A\\
        0 &: x\not\in A
      \end{cases}
    \end{split}
  \end{equation*}
  Show \(\phi\is\TM{bij}\). First show it's injective.
  \[\phi(A)=\phi(B)\]
  \[\implies\chi_A=\chi_B\] % :'<,'>s/(\(.\))/_\1/g
  \[\implies A=B\implies\phi\is\TM{inj}\] % :'<,'>s/(\(.\))/_\1/g
  Now show it's surjective. \(\forall f\in\{f:X\to\{0,1\}\}\;\exists P\in \PP(X)\st
  \phi(P)=f\).
  \[\T{Let }\inv f(\{0,1\})=\inv F\]
  \[\implies \chi_{\inv F}:\inv F\to\{0,1\}\]
  \[\implies\chi_{\inv F}=\phi(\inv F)\T{ w/ }\phi(\inv F)=f\]
  \[\inv F\subseteq X\implies \inv F\in\PP(X)\implies \phi\is\TM{surj}\]
\end{proof}
\ebox


\bbox
\begin{prop}[The Powerset is Larger than the Set]\label{prop:powerset_big}
  If \(X\is\TM{set}\), then \(|X|<|\PP(X)|\).
\end{prop}
\ebox

\bboxproof
\begin{proof}
  Show \(|X|\le|\PP(X)|\). \(f(x)=\{x\}\is\TM{inj}\implies |X|\le|\PP(X)|\). 

  For the sake on contradiction, assume there is a surjection \(g:X\to\PP(X)\).
  Consider \(B:=\{x\in X:x\not\in g(x)\}\). Hence there must be (by surjectivity of \(g\))
  \(z\in X\st g(z)=B\). Someone has to map to it.
  \[z\in B\implies z\not\in g(z)=B\]
  \[z\not\in B\implies z\in g(z)=B\]
  This is a contradiction. So \(|X|<|\PP(X)|\).
\end{proof}
\ebox

\bboxnote\label{note:infinite_infinities}
\TB{Infinite Infinities}. \(|\N|<|\PP(\N)|<|\PP(\PP(\N))|<\dots\)
\ebox

\bbox
\begin{prop}[The Natural Powerset Cardinal is the Continuum Cardinal]\label{prop:abs_pp_n_is_abs_r}
  \(|\PP(\N)|=c\) \((\equiv 2^{\aleph_0}=c\equiv|\{0,1\}|^{|\N|}=|\R|)\)
\end{prop}
\ebox

\bboxproof
\begin{proof}\(\,\)
  \bboxnote
  We'll use the continuum hypthesis, however there's an alternative proof in 
  the course notes.
  \ebox
  Consider \(X=\{f:\N\to\{0,1\}\}\).
  \[\phi:X\to\R:\phi(f):=0.f(1)f(2)f(3)\dots\]
  We can see that \(\phi\) is injective. So
  \[2^{\aleph_0}=|X|\le|\R|=c\]
  Also, \(\aleph_0<2^{\aleph_0}\le c\). So by (CH), we know \(2^{\aleph_0}=c\).
\end{proof}
\ebox


\bboxproof
\begin{proof}[Proof (without (CH))]
  \(\dots\)
\end{proof}
\ebox

\subsection{Cardinal Arithmetic}

\bbox
\begin{defn}
  \(A,B\is\TM{sets}\)
  \begin{enumerate}
    \item \(A\cap B=\emptyset\implies|A|+|B|:=|A\cup B|\)
    \item \(|A|\cdot|B|:=|A\times B|\)
    \item \(|A|^{|B|}:=|\{f:B\to A\}|\)
  \end{enumerate}
\end{defn}
\ebox

\bboxex
\TB{Example.} \(\aleph_0+\aleph_0=\aleph_0\). Let \(A=\{a_1,\dots\}\),
\(B=\{b_1,\dots\}\), so that \(|A|=|B|=\aleph_0\), and \(A\cap B=\emptyset\).

Then \(\phi:A\cup B\to\N:
\phi(a_i):=2i,\;
\phi(b_i):=2i-1\).
This is a bijection.
Hence
\(|A\cup B|=\aleph_0\).
\ebox

\bboxex
\TB{Example.} \(\aleph_0+c=c\).

\(\aleph_0=|\N|,\;|(0,1)|=c\).
\[(0,1)\subseteq\N\cup(0,1)
\subseteq\R\]
\[\implies c\le\aleph_0+c\le c\]
\[\implies \aleph_0+c=c\]
\ebox


\bbox
\begin{prop}[Cardinal Exponent Laws]
  \(A,B,C\is\TM{sets}\).
  \begin{enumerate}
    \item \((|A|^{|B|})^{|C|}
      = |A|^{|B|\cdot|C|}\)
    \item \((|A|^{|B|})(|A|^{|C|})
      = |A|^{|B|+|C|}\)
  \end{enumerate}
\end{prop}
\ebox

\bboxex
\TB{Example.} Show that \(c\cdot c=c\).
\[c\cdot c=(2^{\aleph_0})(2^{\aleph_0})
=2^{\aleph_0+\aleph_0}=2^{\aleph_0}=c\]
\ebox

% lecture 5


\bboxproof
\begin{proof}[Proof of 2.]
  We must show
  \[|\{f|f:B\cup C\to A\}|
  =|\{f|f:B\cup A\}\times\{f|f:B\to A\}|\]
  \[\T{Let }X:=\{f|f:B\to A\}\]
  \[\T{Let }Y:=\{f|f:C\to A\}\]
  \[\T{Let }Z:=\{f|f:B\cup C\to A\}\]
  So, equivically we need to show \(|Z|=|X\times Y|\).
  
  Consider \(\varphi(f,g)(x)=\begin{cases}f(x)&x\in B\\ g(x)&y\in C\end{cases}\).
  \[\varphi(f_1,g_1)=\varphi(f_2,g_2)\]
  \[\implies\forall x\in B\cup C,\;
  \varphi(f_1,g_1)(x)=\varphi(f_2,g_2)(x)\]
  \[\implies\forall x\in B,\;f_1(x)=f_2(x)\implies f_1=f_2\]
  \[\implies\forall x\in C,\;g_1(x)=g_2(x)\implies g_1=g_2\]

  Consider \(h:B\cup C\to A\). Let \(f=h|_B,\;g=h|_C\). Then
  \(\varphi(f,g)=h\).

  So \(\varphi\) is bijective, so proposition 2 holds.
\end{proof}
\ebox


\bboxex
\TB{Example:} \(c^{\aleph_0}=(2^{\aleph_0})^{\aleph_0}
=2^{\aleph_0\cdot\aleph_0}=2^{\aleph_0}=c\).
\ebox

\section{Topology}
\subsection{Metric Spaces}

\bbox
\begin{defn}[Metric Space]\label{defn:metric_space}
  \(X\is\TM{set}\). A metric on \(X\) is a function
  \(d:X\times X\to[0,\infty)\st\)
  \begin{enumerate}
    \item \(d(x,y)=0\iff x=y\)
    \item \TB{Abelian:} \(d(x,y)=d(y,x)\)
    \item \TB{Triangle:} \(d(x,y)\le d(x,z)+d(z,y)\)
  \end{enumerate}
\end{defn}
\ebox


\bbox
\begin{defn}[Normed Vector Space (NVS)]\label{defn:nvs}
  Let \(V\is\TM{Vector\;Space}\) over \(\R\).
  A norm on \(V\) is a fn \(\|\cdot\|:V\to[0,\infty)\st\)
  \begin{enumerate}
    \item \(\|v\|=0\iff v=\vec 0\)
    \item \(\|\alpha v\|=|\alpha|\cdot\|v\|\)
    \item \(\|v+u\|\le\|v\|+\|u\|\)
  \end{enumerate}
\end{defn}
\ebox
\bboxex
\TB{BQN:} \(\|\times==|\circ l\cdot\|\circ r\quad\)\(|+\le\le+\Box|\)
\ebox


\bbox
\begin{prop}[NVS have trivial Metrics]\label{prop:nvs_have_trivial_metrics}
  Let \(V,\|\cdot\|\is\TM{NVS}\). \(d(v,w)=\|v-w\|\) is a metric on \(V\).
\end{prop}
\ebox

\subsection{Examples of Metric Spaces}


\bboxex
\begin{exam}[Discrete Metric]\label{exam:discrete_metric}
  \(X\is\TM{set}\).
  \[
    d(x,y)=\begin{cases}
    0&x=y\\ 1&x\neq y\end{cases}
  \]
\end{exam}
\ebox

\bboxex
\begin{exam}[Absolute Value Norm]\label{exam:absolute_value_norm}
  \((\R,|\cdot|)\is\TM{NVS}\)
\end{exam}
\ebox


\bboxex
\begin{exam}[Euclidean Norm]\label{exam:euclidean_norm}
  \((\R^n,\|\cdot\|_2)\is\TM{NVS}\) where \(\|x\|_2=\sqrt{\sum_{i=1}^n x_i^2}\).
\end{exam}
\ebox

\bboxex
\begin{exam}[P-Norm]\label{exam:p_norm}
  \(p\ge 1,\;(\R^n,\|\cdot\|_p)\is\TM{NVS}\) where
  \[\|x\|_p=\left(\sum_{i=1}^n|x_i|^p\right)^\frac1p\]
  \bboxnote
  \TB{Note:} see posted notes for the proof that this is a norm. OPTIONAL.
  \ebox
\end{exam}
\ebox

\bboxex
\begin{exam}[Infinity Norm]\label{exam:infinity_norm}
  \(p=\infty,\;(\R^n,\|\cdot\|_\infty)\is\TM{NVS}\) where
  \[\|x\|_\infty=\TR{max}\{|x_1|,\dots,|x_n|\}\]
\end{exam}
\ebox

\bboxex
\begin{exam}[P-Norm on Sequences of Reals]\label{exam:p_norm_on_sequences_of_reals}
  \(\R^\N:=\{f|f:\N\to\R\}=\{(a_n)_{n=1}^\infty:a_n\in\R\}\). For \(p\ge 1\),
  \begin{equation}
    \|x\|_p=\left(\sum_{n=1}^\infty|x_n|^p\right)^{\frac1p}
  \end{equation}

  \(l^p:=\{x\in\R:\|x\|_p<\infty\}\implies(l^p,\|\cdot\|_p)\is\TM{NVS}\). This
  is the p-norm on sequences of reals. Notice how this solve the divergence
  issue (by ignoring it lol).
\end{exam}
\ebox


\bboxex
\TB{Example:} \(l^1=\{x\in\R:\sum|x_i|<\infty\}\implies l^p\) is the set of
absolutly convergent sequences.
\ebox


\bboxex
\begin{exam}[Suprema Norm (Infinity Norm on Sequences of Reals)]\label{exam:suprema_norm}
  \(\|x\|_\infty=\sup\{|x_n|:n\in\N\}\).

  if we let \(l^\infty:=\{x\in\R^\N:\|x\|_\infty<\infty\}\), noting that
  \(l^\infty\) is the set of all bounded sequences, then 
  \((l^\infty,\|\cdot\|_\infty)\is\TM{NVS}\).
\end{exam}
\ebox

\bboxex
\begin{exam}[P-Norm on Function]\label{exam:p_norm_on_functions}
  \(C([a,b]):=\{f:[a,b]\to\R|f\is\TM{cts}\}\).
  \[\|f\|_p=\left(\int_a^b|f(x)|\md x\right)^{\frac1p},\;p\ge 1\]
\end{exam}
\ebox


\bboxex
\begin{exam}[Infinity Norm on Functions]\label{exam:infinity_norm_on_functions}
  \(\|f\|_\infty=\sup\{|f(x)|:x\in[a,b]\}\) 
\end{exam}
\ebox


\bboxex
\begin{exam}[Bounded Functions and the Infinity Norm are a NVS]\label{exam:bounded_functions_and_the_infinity_norm_are_a_nvs}
  \(\B([a,b])=\{f:[a,b]\to\R|f\is\TM{bd}\}\), \((\B([a,b]),\|\cdot\|_\infty)\is\TM{NVS}\).
\end{exam}
\ebox

% lecture 6

\bboxex
\begin{exam}[Sequence Metric]
  \(X=\R^\N=\{f|f:\N\to\R\}\). 
  \[d(x,y)=\sum_{i=1}^\infty\frac{|x_i-y_i|}{2^i(1+|x_i-y_i|)}\]
  \bboxproof
  \begin{proof}[Prove that d isn't induced by a metric.]
    If \(d(x,y)=\|x-y\|\) for some norm, then \(\|\alpha x-\alpha y\|
    =|\alpha|\|x-y\|\).
    \begin{align*}
      d(ax,ay)&=\sum_{i=1}^\infty\frac{|ax_i-ay_i|}{2^i(1+|ax_i-ay_i|)}\\
      d(ax,ay)&=\sum_{i=1}^\infty\frac{|a||x_i-y_i|}{2^i(1+|a||x_i-y_i|)}\\
      |a|d(x,y)&=|a|\sum_{i=1}^\infty\frac{|x_i-y_i|}{2^i(1+|x_i-y_i|)}\\
      |a|d(x,y)&=\sum_{i=1}^\infty\frac{|a||x_i-y_i|}{2^i(1+|x_i-y_i|)}\\
      \T{b/c }|a||x_i-y_i|&\neq |x_i-y_i|\\
                          &\implies\T{not induced by a norm}
    \end{align*}
  \end{proof}
  \ebox
\end{exam}
\ebox

\bboxex
\begin{exam}[Cantor Space]\label{exam:cantor_space}
  \(X=2^\N:=\{f|f:\N\to\{0,1\}\}\). 
  \[d(x,y)=\sum_{i=1}^\infty\frac{|x_i-y_i|}{2^i}\]
\end{exam}
\ebox

\bboxex
\begin{exam}[Hamming Distance]\label{exam:hamming_distace}
  \(X\is\TM{finite}\). \(A,B\in\PP(X)\). 
  \[d(A,B):=|A\triangle B|=|(A\cup B)\backslash(A\cap B)|\]
\end{exam}
\ebox

\bboxex
\begin{exam}[Hausdorff Metric]
  \(\HH=\{K\subseteq\R^n:K\T{ compact}\}\). Let 
  \(a\in A,\;b\in B,\;A,B\in\HH\).
  \[d(a,B)=\min\{\|a-b\|:b\in B\}\]
  \[d(b,A)=\min\{\|a-b\|:a\in A\}\]
  \[d(A,B)=\max\{\sup_{a\in A}d(a,B),\sup_{b\in B}d(b,A)\}\]

  \bboxnote
  \begin{note}
    \(\sup_{a\in A}d(a,B)\) represents the biggest shortest path between
    \(A\) and \(B\).
  \end{note}
  \ebox
\end{exam}
\ebox

\bboxnote
\begin{note}
  Metrics give a sense of convergence on a space.
\end{note}
\ebox

\bboxex
\begin{exam}[P-adic Metric]
  Let \(p\) be prime, \(X=\Q\).
  \[\T{Let }0\neq q\in X=\Q\quad q=p^a\frac nm\]
  \[\T{where }\gcd(n,m)=\gcd(p,n)=\gcd(p,m)=1\]
  \[|q|_p=\frac1{p^a},\;|0|_p=0\]
  \[d(q_1,q_2:=|q_1-q_2|_p\]
  \bboxnote
  \begin{note}
    This notion of distance implies the more factors of p, the closer.

    This gives a sense of optimizing for a certain adjective.

    These numbers aren't close using \(\|\cdot\|_2\), but are using
    the p-adic norm.
  \end{note}
  \ebox
\end{exam}
\ebox


\bbox
\begin{defn}[Subspace of a Metric Space]\label{defn:subspace_of_a_metric_space}
  \((X,d),\;Y\subseteq X\implies(Y,d)\). \((Y,d)\) is called a 
  \TB{subspace} of \((X,d)\).
\end{defn}
\ebox

\bbox
\begin{defn}
  \((X,d_1),\;(Y,d_2)\). Consider \((X\times Y,d)\) with
  \begin{align*}
    d((x_1,y_1),(x_2,y_2))&:=d_1(x_1,x_2)+d_2(y_1,y_2)\ (\T{1-norm})\\
    \T{or }d((x_1,y_1),(x_2,y_2))&:=\max\{d_1(x_1,x_2),d_2(y_1,y_2)\}\ (\infty\T{-norm})\\
  \end{align*}
\end{defn}
\ebox

\bboxex
\begin{exam}[Product Metric]\label{exam:product_metric}
  \((X_i,d_i)\;i\in\N\), \(X:=\prod_{n=1}^\infty X_i\). Let 
  \(x=(x_1,x_2,\dots)\T{ and }y=(y_1,y_2,\dots)\).
  \[d(x,y)=\sum_{i=1}^\infty\frac{d_i(x_i,y_i)}{2^i(1+d_i(x_i,y_i))}\]
\end{exam}
\ebox

\subsection{Convergence}

\bbox
\begin{defn}[Convergence of a Sequence]\label{defn:convergence_of_a_sequence}
  \((X,d),\;(x_n)\subseteq X\), and \(x\in X\).
  \bboxnote
  \begin{nota}
    \((x_n)\subseteq X\) means \((x_n)\) is a sequence in \(X\). 
  \end{nota}
  \ebox
  \[(x_n)\T{ conv to }x,\;x_n\to x\TB{ iff}\]
  \[\forall\e>0,\;\exists N\in\N\st\forall n\ge N,\;d(x_n,x)<\e\]
\end{defn}
\ebox


\bboxdefn
\begin{defn}[Divergence]\label{defn:divergence}
  \((x_n)\) diverges of \(\not\exists x\in X\st x_n\to x\).
\end{defn}
\ebox

\bboxnote
\begin{note}[Convergence is Distance going to Zero]
  \((X,d),\;(x_n)\in X,\;x\in x\). Then \(x_n\to x\) \TB{iff}
  \(d(x_n,x)\to 0\).
\end{note}
\ebox

\bboxdefn
\begin{defn}[Cauchy]
  \((X,d)\). \((x_n)\subseteq X\) is a cauchy seq \TB{iff}
  \[\forall\e>0,\;\exists N\in\N\st\forall n,m\ge N,\;
  d(x_n,x_m)<\e\]
\end{defn}
\ebox


\bboxprop
\begin{prop}[Convergence implies Cauchyness]
  \((X,d)\). If \((x_n)\subseteq X\) converges, then \((x_n)\) is cauchy.
\end{prop}
\ebox

\bboxproof
\begin{proof}[Epsilon/2]
  Suppose \((x_n)\to x\). Let \(\e>0\). So \(\exists N\st\forall n\ge N\), 
  \[d(x_n,x)<\gamma\]
  \begin{align*}
    d(x_n,x_m)&\le d(x_n,x)+d(x,x_m)\T{ by }\triangle\\
              &<\gamma+\gamma=2\gamma\\
              &:=2\frac\e2\\
              &=\e
  \end{align*}
\end{proof}
\ebox

\bboxexam
\begin{exam}[Cauchy doesn't imply Convergence]
  \(X=(0,1]\) with the std metric.
  \[\frac1n\to0\implies\left(\frac1n\right)\subseteq X\T{ is cauchy}\]
  \bboxnote
  \begin{note}
    I think this is trying to say that 1/n is cauchy in X, but 
    1/n \(\to\) 0, which is not in X, so it diverges (in X).
  \end{note}
  \ebox

\end{exam}
\ebox

% lecture 7


\bboxdefn
\begin{defn}[Bounded]\ \label{defn:bounded}
    \begin{enumerate}
        \item \(A\subseteq X\) is bd \TU{iff}
            \(\sup\{d(x,y):x,y\in A\}<\infty\)
        \item \((x_n)\subseteq X\is\TM{bd}\iff\{x_1,x_2,\dots\}\is\TM{bd}
            \iff\sup\{n,m\in\N:d(x_n,x_m)\}<\infty\)
    \end{enumerate}
\end{defn}
\ebox

\bboxdefn
\begin{defn}[Open and Closed Balls]\ \label{defn:open_clsd_balls}
    \begin{enumerate}
        \item \TB{Open:} \(B_r(a):=\{x\in X:d(x,a)<r\}\)
        \item \TB{Clsd:} \(B_r[a]:=\{x\in X:d(x,a)\le r\}\)
    \end{enumerate}
\end{defn}
\ebox


\bboxprop
\begin{prop}[Boundedness iff subset of a Closed Ball]\label{prop:bounded_iff_subset_of_closed_ball}
    \((X,d),\;A\subseteq X\). Then \(A\) is bd \TU{iff}
    \(\exists r>0,\;\exists x\in X\st A\subseteq B_r[x]\)
\end{prop}
\ebox

\bboxproof
\begin{proof}
    Suppose \(\sup\{d(x,y):x,y\in A\}=r<\infty\). Assume \(A\neq\emptyset\), taking \(a\in A\).
    For \(b\in A\), \(d(a,b)\le r\implies A\subseteq B_r[a]\).

    Assume \(A\subseteq B_r[a]\implies\forall a,b\in A,\;
    d(a,b)\le d(a,x)+d(x,b)\le 2r\)
\end{proof}
\ebox

\bboxprop
\begin{prop}[Cauchy implies Bounded]\label{prop:cauchy_implies_bounded}
    \((x_n)\is\TM{cauchy}\implies(x_n)\is\TM{bd}\)
\end{prop}
\ebox

\bboxexam
\begin{exam}[Counter Example]
    \((0,1,0,1,0,\dots)\) is bounded but not cauchy.
\end{exam}
\ebox

\bboxnote
\begin{note}
    CONVERGENCE \(\implies\) CAUCHY \(\implies\) BOUNDED
\end{note}
\ebox


\bboxproof
\begin{proof}
    Suppose \((x_n)\) is cauchy.
    Let \(\e=1\).
    \[\exists N\in\N\st
    \forall n,m\ge N\;
    d(x_n,x_m)<1\]
    Let \(r:=\TR{max}\{d(x_1,x_N),\dots,d(x_{N-1},x_N)\}\)
    \bboxnote
    \begin{note}
        We did this because the web's edges are no longer than 1. So, we can look at the 
        \TU{finite} part left. Finite sets are bounded, and the "web" is bounded.
    \end{note}
    \ebox
    \[(x_n)\subseteq B_r[x_N]\]
\end{proof}
\ebox

\subsection{Convergence Examples}


\bboxexam
\begin{exam}[2-Adic Norm]
    Consider \((\Q,|\cdot|_2)\). Let \(x_n:=\frac{1+2^n}3\).
    \bboxnote
    \begin{note}[P-Adic Convergence Claims]
        Looking at \(x_n=\frac13+\frac{2^n}3=\frac13+\cancelto0{\frac{2^n}3}=\frac13\).
    \end{note}
    \ebox
    We claim \(x_n\to\frac13\).
    \bboxproof
    \begin{proof}\
        \begin{align*}
            \left|x_n-\frac13\right|_2&=\left|\frac{2^n}3\right|_2\\
                           &=\frac1{2^n}\\
                           &\to 0
        \end{align*}
        So \(x_n\to\frac13\) under \(|\cdot|_2\).
    \end{proof}
    \ebox
\end{exam}
\ebox

\bboxexam
\begin{exam}[Bounded Sequences and the Infinity Norm]
    \((l^\infty,\|\cdot\|_\infty)\).

    Let \(x_n:=\left(1,\frac12,\frac13,\dots,\frac1n,0,0,0,\dots\right)\) and
    \(x:=\left(1,\frac12,\frac13,\dots\right)\)

    We claim that \(x_n\to x\).
    \bboxproof
    \begin{proof}
        \[\|x_n-x\|_\infty\]
        \[=\sup\left(0,0,\dots,0,\frac1{n+1},\frac1{n+2},\dots\right)\]
        \[=\frac1{n+1}\to0\]
    \end{proof}
    \ebox
\end{exam}
\ebox

\bboxexam
\begin{exam}[Zero Tailed Sequences aren't Cauchy under Sup Norm]\
    \[y_n:=(\underset n{\underbrace{1,1,\dots,1}},0,0,0,\dots)\]
    \[y:=(1,1,1,1,\dots)\]
    \[n\neq m,\;
    \|y_n-y_m\|_\infty=1\]
\end{exam}
\ebox

\subsection{Completeness}


\bboxdefn
\begin{defn}[Complete, Complete Metric Space, Banach Space]\label{defn:complete_complete_metric_banach}
    \((X,d),\;A\subseteq X\). Then 
    \begin{enumerate}
        \item \(A\) is \TB{complete} \TU{iff} every cauchy seq in \(A\) 
            converges to some \(a\in A\).
        \item if \(X\) is complete, we call it a \TB{Complete Metric Space}.
        \item A complete normed vector space is called a \TB{Banach Space}
    \end{enumerate}
\end{defn}
\ebox

\bboxexam
\begin{exam}
    \[X=(0,1],\;\frac1n\to0\not\in X\implies\TM{div}\implies X\not\is\TM{comp}\]
\end{exam}
\ebox

\bboxexam
\begin{exam}
    \[
        A=[1/2,1]\subseteq X\is\TM{comp}
    \]
\end{exam}
\ebox

\bboxexam
\begin{exam}
    \((X,d:=\TM{discrete})\) Let \((x_n)\in\R^\N\) be cauchy.
    \[\exists N\in\N\st\forall n,m\ge N\implies d(x_n,x_m)<1\]
    \[\implies d(x_n,x_m)=0\]
    \[\implies x_n=x_m\]
    \[\implies x_n=(x_1,x_2,\dots,x_N,x_N,x_N,\dots)\to x_N\]
    \[\implies X\is\TM{complete}\]
\end{exam}
\ebox

\bboxnote
\begin{note}
    So nice sets or nice metrics can cause completeness.
\end{note}
\ebox

\bboxexam
\begin{exam}
    Show \((l^\infty,\|\cdot\|_\infty)\) is a Banach Space.

    Let \((x_n)\subseteq l^\infty\), \cref{exam:suprema_norm}. We know already
    that \((l^\infty,\|\cdot\|_\infty)\) is a (NVS), so we have to
    show it's complete. Let \(\e>0\) be given. Then,
    \[\exists N\in\N\st n,m\ge N\implies\|x_n-x_m\|_\infty<\e\]
    \[x_k=(x_k[1],x_k[2],\dots)\]
    \[\T{for }n,m\ge N,\;|x_n[i]-x_m[i]|\]
    \[\le\sup\{|x_n[i]-x_m[i]|:i\in\N\}\]
    \[=\|x_n-x_m\|_\infty<\e\]
    \[\implies\forall i\in\N,\;\T{the seq }(x_n[i])_{n=1}^\infty\is\TM{cauchy\ in\ \R}\]
    \[\R\is\TM{comp}\implies x_n[i]\underset n\to b_i\]
    \[\T{Claim: }x_n\to b:=(b_1,b_2,\dots)\]
    \[
        \forall n,m\ge N,\;
        |x_n[i]-x_m[i]|<\e
    \]
    \[
        \implies\lim_{m\to\infty}|x_n[i]-x_m[i]|\le\e
    \]
    \[
        \implies\forall n\ge N,\;
        |x_n[i]-b_i|\le\e
    \]
    \[
        \T{Consider }\|x_n-b\|_\infty
    \]
    \[
        =\sup\{
            |x_n[i]-b_i|:i\in\N
        \}
    \]
    \[
        \le\e<2\e
    \]
    Hence \(x_n\to b\).

    \TB{Note:} we have that \(x_N-b\in l^\infty\), and \(x_N\in l^\infty\). However
    \(l^\infty\) is a (VS), so \(b\in l^\infty\).
\end{exam}
\ebox


\bboxprop
\begin{prop}[Set of bounded Sequences on the P-Norm is Banach]
    \((\ell^p,\|\cdot\|_p)\is\TM{banach}\).
    \[
        \|x\|_p=\left(\sum_{n=1}^\infty|x_n|^p\right)^{\frac1p}
    \]
    \[
        \ell^p:=\{x\in\R:\|x\|_p<\infty\}\implies(\ell^p,\|\cdot\|_p)\is\TM{NVS}
    \]
\end{prop}
\ebox

\bboxproof
\begin{proof}
    Let \((a_k)\subseteq\ell^p\) be cauchy.
    \[
        \T{Say }a_k=(a_k[1],a_k[2],\dots)
    \]
    \[
        \T{Let }\e>0
    \]
    \[
        \exists N\in\N\st\|a_k-a_m\|<\e,\;
        \forall k,m\ge N
    \]
    \[
        \T{Fixing }i\in\N,\;
        \T{Since }|a_k[i]-a_m[i]|\le\|a_k-a_m\|_p<\e
    \]
    \[
        \T{We see that }(a_k[i])_{k=1}^\infty\is\TM{cauchy\ in\ \R}
    \]
    \[
        \R\is\TM{comp}\implies a_k[i]\to b_i\T{ for some }b_i\in\R
    \]
    \[
        \T{Claim: }a_k\to b=(b_1,b_2,\dots)
    \]
    \[
        \forall k,m\ge N,\T{ we see that}
    \]
    \[
        \sum_{i=1}^M|a_k[i]-a_m[i]|^p
    \]
    \[
        \le\sum_{i=1}^\infty|a_k[i]-a_m[i]|^p
    \]
    \[
        =\|a_k-a_m\|_p^p<\e^p
    \]
    \[
        \sum^M_{i=1}|a_k[i]-b_i|^p\le\e^p,\;\forall M\in\N
    \]
    \[
        M\to\infty:\sum_{i=1}^\infty|a_k[i]=b_i|^p\le\e^p
    \]
    \[
        \implies\|a_k-b\|_p\le\e,\;\forall k\ge N
    \]

    Noting that \(a_N,a_N-b\in\ell^p\implies b\in\ell^p\).
\end{proof}
\ebox

\bboxexam
\begin{exam}\
    \[
        C_{00}=\{(x_n)\in\ell^\infty:\exists N\in\N\st\forall n\ge N,\;x_n=0\}
    \]
    \[
        \is\TM{NVS},\;\T{via }\|\cdot\|_\infty
    \]

    Consider
    \[
        x_n=(1,1/2,\dots,1/n,0,0,0,\dots)
    \]
    \[
        x_n\is\TM{cauchy,\ divergence}\T{ b/c}
    \]
    \[
        x_n\to(1,1/2,1/3,\dots)\not\in C_{00}\implies C_{00}\is\TM{\neg comp}
    \]
\end{exam}
\ebox

% lecture 8
\subsection{Topological Spaces}


\bboxdefn
\begin{defn}[Topology]
    \(X\) set. A \TB{topology} on \(X\) is a set \(\TT\subseteq\PP(X)\) s/t
    \begin{enumerate}
        \item \(\emptyset,X\in\TT\)
        \item \(U,V\in\TT\implies U\cap V\in\TT\)
        \item \(U_i\in\TT,\;(i\in I)\implies\bigcup_{i\in I}U_i\in\TT\)
    \end{enumerate}
\end{defn}
\ebox

\bboxexam
\begin{exam}[Discrete Topology]
    \(X\) set. \(\TT:=\PP(X)\)
\end{exam}
\ebox

\bboxexam
\begin{exam}[Indiscrete Topology]
    \(\TT:=\{\emptyset,X\}\)
\end{exam}
\ebox

\bboxexam
\begin{exam}
    \(X=\{a,b,c\}\)
    \[
        \TT_1=\{\emptyset,X,\{a\},\{b,c\}\}
    \]
    \[
        \TT_2=\{\emptyset,X,\{a\},\{b\},\{a,b\}\}
    \]
\end{exam}
\ebox

\bboxnota
\begin{nota}[Topological Space]
    If \(\TT\) is a topology on \(X\), then \((X,\TT)\) is called a
    \TB{topological space}.
\end{nota}
\ebox

\subsection{Metric Topology}

\bboxdefn
\begin{defn}[Open]
    \((X,\TT)\). \(U\subseteq X\) is open \TU{iff}
    \(\forall x\in U,\;\exists r>0\st B_r(x)\subseteq U\).
\end{defn}
\ebox

\bboxprop
\begin{prop}[The Set of Open sets form a Topology]
    \(\TT=\{U\subseteq X:U\T{ open}\}\) is a topology.
\end{prop}
\ebox

\bboxproof
\begin{proof}
    \(\emptyset,X\subseteq\TT\) trivially.

    Let \(U,V\in\TT\). Since \(U\mathbin{\&}V\) are open
    \[
        \exists r_1,r_2>0\st\quad
        B_{r_1}(x)\subseteq U\quad B_{r_2}(x)\subseteq V
    \]
    \[
        r:=\min\{r_1,r_2\}\quad
        B_r(x)\subseteq U\cap V\implies U\cap V\in\TT
    \]

    Let \(U_i\in\TT\) for all \(i\in I\). Let \(x\in\bigcup_{i\in I}U_i\).
    \[
        \T{So }\exists i\in I\st x\in U_i
    \]
    \[
        \T{So }\exists r>0\st B_r(x)\subseteq U_i\subseteq\bigcup U_i
    \]
\end{proof}
\ebox

\bboxexam
\begin{exam}[Counterexample for Infinte Intersections are Open]
    \[
        \left(\R,\TT_{\TR{open}}\right)\quad U_n=\left(-\frac1n,\frac1n\right)
        \quad\bigcap U_n=\{0\}\not\in\TT
    \]
\end{exam}
\ebox

\bboxprop
\begin{prop}[Metric Spaces are Haussdorff]
    \((X,d)\ \forall x\neq y\in X,\;\exists U,V\subseteq X\T{ open}\st x\in U,\;y\in V,\;
    U\cap V=\emptyset\)
\end{prop}
\ebox

\bboxproof
\begin{proof}[Proof]
    Let \(r=d(x,y)>0\). Let \(U=B_{r/2}(x)\ V=B_{r/2}(y)\).
    \bboxproof
    \begin{proof}[Open Balls are Open Proof]
        Let \(x\in B_r(a)\) for some \(a\in X,\;r>0\). Then
        \[
            d(x,a)<r\T{ by open ball def}
        \]
        \[
            y\mathbin{:\in}B_{r-d(x,a)}(x)
        \]
        \[
            \implies d(x,y)<r-d(x,a)
        \]
        \[
            d(y,a)\overset{\TM{TI}}<d(x,y)+d(x,a)<r
        \]
        \[
            \implies y\in B_r(a)\implies B_{r-d(x,a)}(x)\subseteq B_r(a)
            \implies B_r(a)\T{ is open}
        \]
    \end{proof}
    \ebox
    Assume for the sake of contradiction \(\exists z\in B_{r/2}(x)\cap B_{r/2}(y)\).
    \[
        \implies d(z,x)<r/2\quad d(z,y)<r/2
    \]
    \[
        r>d(x,z)+d(z,y)>d(x,y)=r
    \]
    This is a contradiction.
\end{proof}
\ebox

\bboxexam
\begin{exam}
    \(X=\{a,b,c\},\;\TT=\{\emptyset,X,\{a,b\}\}\). Since 
    \(a\) cannot be "seperated" from \(b\), then there is 
    no possible metric on \(X\), so \(\TT\) isn't a metric
    topology.

    All metrics make a topology, not all topologies make a metric.
\end{exam}
\ebox

\subsection{Closed Sets}

\bboxdefn
\begin{defn}[Closed Sets]
    \((X,\TT)\).
    \(C\subseteq X\) is closed \TU{iff} \(X\backslash C\in\TT\)
\end{defn}
\ebox

\bboxprop
\begin{prop}[Properties of Closed Sets]
    \((X,\TT)\).
    \begin{enumerate}
        \item \(\emptyset,X\) closed.
        \item \(C,D\) closed then \(C\unin D\) closed.
        \item \(C_i,\;i\in I,\;\bigintr_{i\in I}C_i\) is closed.
    \end{enumerate}
\end{prop}
\ebox


\bboxproof
\begin{proof}[Proof by "Boolean Nonsense"]
    \(X\smin C,\;X\smin D\in\TT\).
    \[
        X\smin(C\unin D)\in\TT
    \]
    \[
        X\smin C\intr X\smin D\in\TT
    \]
\end{proof}
\ebox

\bboxdefn
\begin{defn}[Limit Point]
    \((X,\TT),\;A\ssq X\).
    We say \(x\in X\) is a \TB{limit point} of \(A\) iff
    \[
        \forall U\in\TT\T{ with }x\in U\quad A\intr U\neq\emptyset
    \]
\end{defn}
\ebox

\bboxnote
\begin{note}
    \(x\in A\implies x\is\TM{limit\ point}\)
\end{note}
\ebox

\bboxprop
\begin{prop}[Limits Points in a Metric Topology are the Limit of a Sequence]
    \(A\in X\). Then \(x\in X\) is a limit point of \(A\) \TU{iff}
    \(\exists(a_n)\in A\st a_n\to x\).
\end{prop}
\ebox

\bboxproof
\begin{proof}[\((\implies)\) Proof]
    Assume \(x\) is a limit a point of \(A\). Then
    \(\forall U\in\TT,\;x\in U,\;A\intr U\neq\emptyset\).
    Then \(\forall n\in\N,\;\exists a_n\in B_{\frac1n}(x)\intr A\quad(\neq\emptyset)\).
    Then \(d(x,a_n)<\frac1n\to0\implies a_n\to x\)
\end{proof}
\ebox

\bboxproof
\begin{proof}[\((\limplies)\) Proof]
    Assume \(\exists a_n\to x\). Then \(\forall\e>0\ \exists N\in\N\st d(a_n,x)<\e\).
    Let \(U\subseteq X\) be open with \(x\in U\).
    \[
        \implies\exists r>0\st B_r(x)\ssq U
    \]
    \[
        \T{Then }\exists N\in\N\st d(a_N,x)<r
    \]
    \[
        \T{So }a_N\in U\implies A\intr U\ni a_N
    \]
\end{proof}
\ebox


\bboxprop
\begin{prop}[Closed Sets Attain Limits]
    C is closed \TU{iff}
    C attains its limits points.
\end{prop}
\ebox


\bboxproof
\begin{proof}[\((\implies)\) Proof]
    Suppose C is clsd.
    Let \(x\in X\) be a limit point of C.
    Since \(X\smin C\) is open
    and \((X\smin C)\intr C=\emptyset\),
    \(x\not\in X\smin C\implies x\in C\).
\end{proof}
\ebox

\bboxproof
\begin{proof}[\((\limplies)\) Proof]
    Suppose C attains its limit points.
    Show \(X\smin C\in\TT\).
    \[
        \llet x\in X\smin C
    \]
    \[
        \T{So }x\T{ isn't a limit point of }C
    \]
    \[
        \tthen\exists U_x\in\TT,\;x\in U_x\st
        U_x\intr C=\emptyset
    \]
    \[
        \tthen U_x\ssq U_x\intr C\quad(?)
    \]
    \[
        \tthen X\smin C=
        \bigunin_{x\in X\smin C}U_x
        \in\TT
    \]
\end{proof}
\ebox

\bboxcoro
\begin{coro}[Closed Sets Attain Sequence Limits]\label{coro:closed_sets_attain_sequence_limits}
    \((X,d),\ C\ssq X\). Then \(C\) is closed
    \TU{iff}
    \[
        \forall(c_n)\ssq C,\ c_n\to x\in X
        \tthen x\in C
    \]
\end{coro}
\ebox

% lecture 9

\bboxdefn
\begin{defn}[Subspace Topology]
  \((X,\TT),\;Y\ssq X\).

  \TB{Note }that \(\vphi,X\in\TT,\ \TT\T{ closed under }(\unin,\intr)\)

  The \TB{subspace topology} is 
  \[
  \TT' = \{
  Y \intr U : U \in \TT
  \}
  \]
  \bboxproof
  \begin{proof}[Prove subspace topologies are Topologies]\ 
    \begin{enumerate}
      \item Show
        \[
        \emptyset, Y \in \TT'
        \]
        So,
        \[
        \emptyset\in\TT,\
        Y \intr \emptyset = \emptyset
        \implies \emptyset\in\TT'
        \]
        \[
        X\in\TT,\ Y\intr X=Y
        \implies Y \in \TT'
        \]
      \item Show
        \[
        U,V\in\TT'\implies U\intr V\in\TT'
        \]
        So let \(U,V\in\TT'\).
        \[
        \implies \exists\T{open }U_X\in X\st U=U_X\intr Y
        \]
        \[
        \implies\exists\T{open }V_X\in X\st V=V_X\intr Y
        \]
        \[
        \implies
        U \intr V
        =(U_X\intr Y) \intr (V_X\intr Y)
        =(U_X\intr V_X)\intr Y
        \]
        \[
        U_X\intr V_X\in\TT\implies
        U\intr V\in\TT'
        \]
      \item Show
        \[
        U_i\in\TT',\ (i\in I)\implies\bigunin_{i\in I}U_i\in\TT'
        \]
        So
        \[
        U_i\in\TT'\implies U_i=Y\intr U^X_i\T{ for some open }U^X_i\in\TT
        \]
        \[
        \bigunin U_i=Y\intr\bigunin U^X_i\implies U_i\in\TT'
        \]
    \end{enumerate}
  \end{proof}
  \ebox
\end{defn}
\ebox

\bboxnote
\begin{note}
  \((X,\TT),\ Y\ssq X,\ \TT'\T{ as above}.\ C\ssq Y\T{ clsd}\).
  \[
  \implies Y\smin C \in \TT'
  \]
  \[
  \implies Y\smin C=Y\intr U,\ U\in\TT
  \]
  \[
  \implies C = Y \intr \underset{\T{clsd}}{\underbrace{(X\smin U)}}
  \]
\end{note}
\ebox

\bboxnote
\begin{note}
    \((X,d),\ Y\ssq X\). Define \((Y,d)\) as a subspace metric space.

    Suppose \(U\ssq Y\) is open wrt \(Y\).
    \[
        \implies \forall x\in U\ \exists r_x>0\st
        \underset{\T{in }Y}{\underbrace{B_{r_x}(x)}}\ssq U
    \]
    \[
        \implies \forall x\in U\ \exists r_x>0\st
        \underset{\T{in }X}{\underbrace{
                B_{r_x}(x)
        }} \intr Y \ssq U
    \]
    \[
        U=\bigunin_{x\in U}(Y\intr B_{r_x}(x))
        =Y\intr\underset{\T{open in }X}{\underbrace{
                \left(
                    \bigunin_xB_{r_x}(x)
                    \right)
        }}
    \]
\end{note}
\ebox

\subsection{Closure and Interior}
\bboxdefn
\begin{defn}[Closure and Interior]\ 
    \begin{enumerate}
        \item the \TB{Closure} of \(A\) is defined as follows:
            \[
                \cls A=\underset{C\T{ clsd}}{\bigunin_{C\supseteq A,}}C
            \]
        \item the \TB{Interior} of \(A\) is defined as follows:
            \[
                \interior(A)=\underset{U\in\TT}{
                    \bigunin_{U\ssq A} U
                }
            \]
    \end{enumerate}
\end{defn}
\ebox

\bboxnote
\begin{note}
    \begin{enumerate}
        \item \(\cls A\is\T{clsd}\ \interior(A)\in\TT\)
        \item \(\interior(A)\ssq A\ssq\cls A\)
        \item 
            \[
                A\T{ closed}\iff A=\cls A
            \] \[
                A\T{ open}\iff A=\interior(A)
            \]
    \end{enumerate}
\end{note}
\ebox

\bboxnote
\begin{note}
    \((X,\TT),\ Y\ssq X,\ A\ssq Y.\ (Y,\TT')\).
    \[
        (\T{wrt}\ Y)\longrightarrow\cls A
        = \bigintr\{
            C : A\ssq C,\ C\ \TR{clsd\ in}\ Y
        \}
    \]
    \[
        =\bigintr\{
            Y\intr C : A\ssq C,\ C\ clsd\ in\ X
        \}
    \]
    \[
        =Y\intr\left(
            \bigintr\{
                C : A \ssq C,\ C\ clsd\ in\ X
            \}
        \right)
    \]
    \[
        = Y \intr \cls A\longleftarrow(\T{wrt }X)
    \]
    Similiarly, \(\interior(A)\T{ wrt }(Y)=Y \intr \interior(A)\T{ wrt }(X)\).
\end{note}
\ebox

\bboxprop
\begin{prop}[The Closure is the Set of Limit Points]
    \(A\ssq X\).
    \[
        \cls A = \{
            x \in X : x \T{ is a limit point of }A
        \}
    \]
\end{prop}
\ebox

\bboxproof
\begin{proof}
    Let \(L:=\{
        x \in X : x\T{ is a limit point of }A
    \}\). Let \(x\in\cls A\). Let \(U\in\TT\st x\in U\).
    Suppose \(A\intr U=\emptyset\)
    \[
        \implies A\ssq\underset{clsd}{\underbrace{
                X\smin U
        }}\implies x\in X\smin U
    \]
    Contradiction, \(x\in U\) and \(x\in X\smin U\).

    Let \(x\in L\), and let \(C\) be clsd, with \(A\ssq C\).
    Suppose
    \[
        x\not\in C\implies x\in X\smin C:=U
    \]
    \[
        \implies(X \smin C)\intr A \neq \emptyset
    \]
    Contradiction of \(A\ssq C\).
\end{proof}
\ebox


\bboxnote
\begin{note}[Norms]\ 
    \begin{enumerate}
        \item P-Norm, \(x\in\R^n\).
            \[
                \left(
                    \sum_{i=1}^n|x_i|^p
                \right)^{\frac1p}
            \]
        \item Inf-Norm, \(x\in\R^n\).
            \[
                \max\{|x_i| : 0 \le i \le n\}
            \]
        \item P-Norm, \(x\in\R^\N\)
            \[
                \left(
                    \sum_{i=1}^\infty|x_i|^p
                \right)^{\frac1p}
            \]
        \item Inf-Norm, \(x\in\R^\N\).
            \[
                \sup\{|x_i|:i\in\N\}
            \]
        \item P-Norm, \(x\in\R^\R\).
            \[
                \left(
                    \int_{-\infty}^\infty|f(x)|^p\,\mathrm dx
                \right)^{\frac1p}
            \]
        \item Inf-Norm, \(x\in\R^\R\).
            \[
                \sup\{|f(x)|:x\in\R\}
            \]
        \item \(\ell^p=\{x\in\R^\N : \|x\|_p<\infty\}\)
    \end{enumerate}
\end{note}
\ebox

\bboxcoro
\begin{coro}[Closure is the Set of Reachable Points]
    \(\cls A=\{x\in X : \exists(a_n)\ssq A,\ a_n\to x\}\)
\end{coro}
\ebox

\bboxdefn
\begin{defn}[Interior Points]
    \((X,\TT),\ A\ssq X\). Then \(x\in A\) is an \TB{interior point}
    \TU{iff} 
    \[
        \exists U\in\TT\st x\in U\ssq A
    \]
    \bboxnote
    \begin{note}
        Notice that this is similiar to how we define openness in a metric 
        space.
    \end{note}
    \ebox
\end{defn}
\ebox

\bboxnote
\begin{note}
    \((X,d),\ A\ssq X\). Then \(x\in A\) is an interior point of \(A\) \TU{iff}
    \[
        \exists r > 0 \st x \in B_r(x) \ssq A
    \]
\end{note}
\ebox

\bboxprop
\begin{prop}[Interior Points of A build the Interior of A]
    \[
        \interior(A) = \{
            x \in A : x \is \T{an interior point of }A
        \}
    \]
\end{prop}
\ebox

\bboxproof
\begin{proof}
    Let \(I:=\{ x : x\is\T{int pt}\}\), \(x\mathbin{:\in}\interior(A)\).
    \[
        \iff\exists U\in\TT,\ x\in U\ssq A\iff x\in I
    \]
\end{proof}
\ebox

\bboxexam
\begin{exam}[Closures and Closed ``Versions'' of Sets are Inequal]
    \((\N,|\cdot|)\)
    \[
        B_1(1) = \{1\}\implies \cls{B_1(1)}=\{1\}=B_1(1)
    \]
    \[
        B_1[1]=\{1,2\}
    \]
\end{exam}
\ebox

\bboxproof
\begin{proof}
    \TB{Prove} \(\cls{B_r(x)}=B_r[x]\), under a (NVS).

    NVS: vectors, w/ buildin norm. Basically this is a ordinary norm, 
    rather than a metric. So we have norm TI, norm zero uniqueness,
    AND a scalar property.

    \((\ssq)\). Let \(y\in\cls{B_r(x)}\). So 
    \[
        B_r[x]=\{y\in\R : \|x-y\|\le r\}
    \]
    \[
        \T{Show }\|x-y\|\le r
    \]
    \[
        y\in\cls{B_r(x)}\implies\exists y_n\in\R^\N\st y_n\to y
    \]
    \[
        \implies\exists N\in\N\st\forall n>0,\
        \|x-y_n\|<r
    \]
    \[
        \implies \|x-y\|\le r
    \]
    \((\supseteq)\) Let \(y\in B_r[x]\).
    \[
        \|y-x\|\le r
    \]
    \[
        \T{Show }\exists(a_n)\in\R^\N\st a_n\to y
    \]
    \[
        \llet a_i=y+\frac{x-y}i
    \]
    \[
        \|x-a_i\|=\left\|x-y-\frac{x-y}i\right\|=\|x-x/i-y+y/i\|
    \]
    \[
        \|x-x/i+(y/i-y)\|\le\|x-x/i\|+\|y-y/i\|<2r
    \]
\end{proof}
\ebox

% lecture 10

\bboxexam
\begin{exam}
    \((X,d),\ A\ssq X\T{ complete}\).
    Prove \(A\) is closed.
    \[
        (a_n)\ssq A,\ a_n\to x\in X
    \]
    \[
        (a_n)\T{ cauchy},\
        (a_n)\ \T{conv to a pt in }A.
    \]
    \[
        \implies x \in A\implies \T{closed}
    \]
\end{exam}
\ebox

\bboxdefn
\begin{defn}[Subsequence]
    \[
        (X,d),\ (X_n)\subseteq X
    \]
    A subsequence of \((X_n)\) is a sequence 
    \[
        (X_{n_k})_{k=1}^\infty,\ \TR{where}\ 
        n_1 < n_2 < n_3 < \cdots
    \]
    \bboxnote
    \begin{note}
        \[
            \mathbb{N}\rightarrow\mathbb{N}\rightarrow X : x(n(k)),\ n\T{ increasing}
        \]
        \[
            K\ge N,\ n_K\ge K\ge N
        \]
        \[
            K\ge N,\ n(K)\ge K\ge N
        \]
    \end{note}
    \ebox
\end{defn}
\ebox

% `(eq  \ssq
\bboxexam
\begin{exam}
    \[
        (X,d),\ (X_n)\subseteq X.\ 
        \T{Prove}\ x_n\to x
        \implies x_{n_k}\to x
    \]
    \bboxproof
    \begin{proof}
        \[
            K\ge N,\ n_K\ge K\ge N
        \]
        \[
            \implies d(x_{n_k},x)<\epsilon
        \]
    \end{proof}
    \ebox
\end{exam}
\ebox

\bboxexam
\begin{exam}
    \[
        \llet (X,d)\ (x_n)\ssq X\T{ cauchy}
    \]
    \[
        \T{Show }x_n\T{ conv}\iff x_n \T{ has a conv subseq}
    \]
    \bboxproof
    \begin{proof}
        The forward direction is trivial.

        Backwards: Let \((x_n)\) be cauchy, and let
        some subsequence \(x_{n_k}\to x\).
        \[
            \T{Let}\ \epsilon > 0,
            \T{let}\ N\in\N\st
            \forall n,m\ge N
            \implies d(x_n,d_m)<\frac\e2
        \]
        \[
            \llet\ k\in\N\ k\ge K
            \implies d(x_{n_k},x)<\frac\e2
        \]
        \[
            \T{Assume}\ K\ge N\implies n_K\ge N
        \]
        \[
            \forall n\ge N,\ d(x_n,x)\le
            d(x_n,x_{n_k})+d(x_{n_k},x)
        \]
        \[
            <\frac\e2+\frac\e2=\e
        \]
    \end{proof}
    \ebox
\end{exam}
\ebox

\bboxexam
\begin{exam}
    \[
        \llet\ (X,d),\ x_n\to x
    \]
    \[
        \TR{Prove}\
        C=\{x_n:n\in\N\}\unin\{x\}\ \is\ \TR{closed}
    \]
    \bboxproof
    \begin{proof}
        \[
            \iif\ \exists N\ \TR{inf\ many}\ y_n=x_N
        \]
        \[
            \tthen\ (x_N)_n\is\ \TR{a\ subseq\ of}\ y_n
        \]
        \[
            y_n\to y,\ (x_N)_n\to x_N\implies y=x_N\in C\ \blacksquare
        \]


        \[
            \iif\ \forall n,\ \TR{only\ finitly\ many}\ y_K=x_n
        \]
        \[
            \tthen\ \exists(y_{n_k})
        \]
        which is also a subseq of \((x_n)\). Then \(y=x\). \(\blacksquare\)
    \end{proof}
    \ebox
\end{exam}
\ebox


\bboxexam
\begin{exam}
    Let \(V\) be a (NVS), \(U\ssq V,\ U\in\TT,\ x\in V\). Show
    \[
        x+U=\{x+u\colon u\in U\}\in\TT
    \]

    \bboxproof
    \begin{proof}
        \[
            \llet\ x+u\in x+U,\ u\in U\in\TT
        \]
        \[
            u\in U\in\TT\implies \exists r>0\st
            B_r(u)\ssq U
        \]
        \[
            B_r(x+u)=x+B_r(u)\ssq x+U\implies x+U\in\TT
        \]
    \end{proof}
    \ebox
\end{exam}
\ebox

\bboxexam
\begin{exam}
    \(V\ \is\ \TR{NVS},\ \TT\ni U\ssq V,\ A\ssq V.\)
    \[
        \TR{Show}\ A+U=\{a+u:a\in A,\ u\in U\}\in\TT
    \]
    \bboxproof
    \begin{proof}
        \[
            A+U=\bigunin_{a\in A}a+U
        \]
        \[
            a+U\in\TT\implies A+U\in\TT
        \]
    \end{proof}
    \ebox
\end{exam}
\ebox

% lecture 11


\bboxexam
\begin{exam}
    \[
        V\ \TM{NVS},\ C\ssq V\ \TR{clsd},\ x\in V
    \]
    prove \(x+C\) is closed.
    \bboxproof
    \begin{proof}
        \[
            V\smin C\in\TT\implies x + V\smin C\in\TT\ \TR{from\ above}
        \]
        \[
            V\smin(x+V\smin C)\ \TR{is\ closed\ by\ defn}
        \]
        \[
            =x+C
        \]
        \[
            -C\in\TT.\ x+(-C).\ -(x+(-C))=x+C
        \]
    \end{proof}
    \ebox

\end{exam}
\ebox

\bboxexam
\begin{exam}
    Find \(C,D\ssq \R\) closed s/t \(C+D\) isn't closed.
    \bboxproof
    \begin{proof}
        \(C:=\N,\ D:=\{-n+1/n : n\ge 2\}\).
        \[
            \frac1n\in C+D,\ n\ge 2
        \]
        \[
            0\not\in C+D\implies C+D\neq\cls{C+D}
        \]
    \end{proof}
    \ebox
\end{exam}
\ebox


\bboxexam
\begin{exam}
    \[
        (X,d),\ A\ssq X.\ \TR{Show}\ 
        X\smin\interior(A)=\cls{X\smin A}
    \]
    \bboxproof
    \begin{proof}[(\(\ssq\)) Proof]
        Let \(x\in X\smin\interior(A)\).
        Let \((x_n)\ssq X\smin A\st x_n\in B_{\frac1n}(x)\).
        \[
            x_n\to x\implies X\smin\interior(A)\ssq \cls{X\smin A}
        \]
    \end{proof}
    \ebox

    \bboxproof
    \begin{proof}[\((\supseteq)\) Proof]
        Let \(x\in\cls{X\smin A}\).

        Then \(\exists x_n\ssq X\smin A\st x_n\to x\).

        Then \(\forall\e>0,\ \exists N\in\N\st x_N\in B_\e(x)\).

        Note that since \(x_N\not\in A\ \tthen\ B_\e(x)\not\ssq A\).

        Since \(y\not\in\interior(A)\iff\forall\e>0,\ B_\e(x)\not\ssq A\),
        then \(x\not\in\interior(A)\).
    \end{proof}
    \ebox
\end{exam}
\ebox


\bboxexam
\begin{exam}
    Prove \(X\smin\cls{A} = \interior{X \smin A}\).
    \bboxproof
    \begin{proof}
        We know from above that \(X\smin\interior(X\smin A)
        = \cls{X\smin(X\smin A)}\).

        So \(X\smin\interior(X\smin A) = \cls{A}\).

        Then \(X\smin \cls{A} = \interior(X\smin A)\).
    \end{proof}
    \ebox
\end{exam}
\ebox

\bboxdefn
\begin{defn}[Boundary]
    \((X,d),\ A\ssq X\). The \TB{boundary} of \(A\) is
    \[
        \partial A=\cls{A}\smin\interior(A)
    \]
\end{defn}
\ebox

\bboxexam
\begin{exam}
    Prove \(\partial A\) is closed.
    \bboxproof
    \begin{proof}
        \[
            \partial A=\cls{A}\smin\interior(A)
        \]
        \[
            =\cls{A}\intr(X\smin\interior(A))
        \]
        \[
            =\cls{A}\intr\cls{X\smin A}
        \]
        So \(\partial A\) is closed, because intersections of
        closed sets are closed.
    \end{proof}
    \ebox
\end{exam}
\ebox

\bboxexam
\begin{exam}
    Prove \(A\) is closed \TU{iff} \(\partial A\ssq A\).
    \bboxproof
    \begin{proof}[\((\implies)\) Proof]
        \[
            A\T{ is closed}\implies A=\cls{A}
        \]
        \[
            \partial A=\cls{A}-A^0=A-A^0\ssq A
        \]
    \end{proof}
    \ebox

    \bboxproof
    \begin{proof}[\((\limplies)\) Proof]
        \[
            \partial A\ssq A\implies\cls{A}\smin A^0\ssq A
        \]
        \[
            \implies \cls{A}\ssq A\unin A^0=A
        \]
    \end{proof}
    \ebox
\end{exam}
\ebox

\bboxdefn
\begin{defn}[Hausdorf]
    \((X,\TT)\) is Hausdorf \TU{iff}
    \[
        \forall x\neq y\in X
    \]
    \[
        \exists\ \TR{disjoint}\ U,V\in\TT\st x\in U,\ y\in V.
    \]
\end{defn}
\ebox

\bboxexam
\begin{exam}
    Let \((X,\TT)\) be Hausdorf. Show
    \(\{x\}\) is closed.
    \bboxproof
    \begin{proof}
        
        \[
            \forall y\neq x,\ U_y,V_y\in\TT,\ U_y\intr V_y=\emptyset,\ 
            y\in U_y,\ x\in V_y
        \]
        \[
            X\smin\{x\}=\bigunin_{y\neq x}U_y\in\TT
        \]
    \end{proof}
    \ebox
\end{exam}
\ebox

\bboxexam
\begin{exam}
    \((X,\PP(X))\). Prove \(\TT=\PP(X)\) is induced by a metric.
    \bboxproof
    \begin{proof}
        Let
        \[
            d(x,y)=\delta(x,y):=\begin{cases} 1 & x=y \\ 0 & o/w \end{cases}
        \]
        We need to show \(\forall A\in\TT,\ A\T{ is open}\). Let \(A\in\TT\).
        \[
            A=\bigunin_{a\in A}\{a\}=\bigunin_{a\in A}B_1(a)
        \]
        Since \(B_1(a)\) is open, and arbitrary unions of open sets wrt
        \(\delta\) are open, then \(A\) is open.
    \end{proof}
    \ebox

\end{exam}
\ebox

\section{Continuity}
\subsection{Continuity}

\bboxdefn
\begin{defn}[Topologic Continuity]
    Let \((X,\TT),\ (Y,\TT')\) be topological spaces.

    \(f:X\to Y\) is cts \TU{iff}
    \[
        f^{-1}(U)\in\TT,\ \forall U\in\TT'
    \]

    Noting that \(f^{-1}(U)=\{x\in X:f(x)\in U\}\).
\end{defn}
\ebox

\bboxprop
\begin{prop}[Closedness and Continuity]
    \((X,\TT),\ (Y,\TT'),\ f:X\to Y\). TFAE:
    \begin{enumerate}
        \item \(f\ \TR{cts}\)
        \item \(\forall A\ssq X,\ f(\cls{A})\ssq\cls{f(A)}\)
        \item \(\forall\TR{closed}\ C\ssq Y,\ f^{-1}(C)\ \TR{is\ closed\ in}\ X\)
    \end{enumerate}
\end{prop}
\ebox

\bboxproof
\begin{proof}[\((1)\implies(2)\) Proof]
    Assume \(f\) is cts.

    Let \(y\in f(\cls{A})\).

    Show \(\forall U\in\TT'\st y\in U,\ U\intr f(A)\neq \emptyset\).
    
    Let \(y=f(x),\ \TR{for\ some}\ x\in\cls{A}\).

    So \(\forall V\in\TT\st x\in V,\ V\intr A\neq \emptyset\).

    Let \(U\in\TT'\st y\in U\).

    \(\implies f(x)\in U\).

    \(\implies x\in f^{-1}(U)\), and \(f^{-1}(U)\in\TT\) by continuity.

    Since \(f^{-1}(U)\in\TT,\ x\in f^{-1}(U),\ x\in\cls{A}\),
    then by the defn of the closure,
    \(f^{-1}(U)\intr A\neq \emptyset\).

    Let \(a\in f^{-1}(U)\intr A\neq\emptyset\)

    \(\implies f(a)\in U \intr f(A)\)

    \(\implies y \in \cls{f(A)}\)
\end{proof}
\ebox


\bboxproof
\begin{proof}[\((2)\implies(3)\) Proof]
    \(\llet:C\ssq Y\ \TR{be\ closed\ and}\ A=f^{-1}(C).\)

    \(\TR{Show}\ A\ \is\ \TR{closed}.\ \TR{Show}\ \cls{A}\ssq A.\)

    \(\TR{For}\ x\in\cls{A},\ f(x)\in f(\cls{A})\ssq\cls{f(A)}\ssq \cls{C}=C\)

    \(\implies x\in f^{-1}(C)=A\)

    \(\implies A=\cls{A}\implies A\ \is\ \TR{closed}\)
\end{proof}
\ebox

\bboxproof
\begin{proof}[\((3)\implies(1)\) Proof]
    \(\TR{For}\ U\ssq Y\ \TR{open},\ Y\smin U\ \TR{clsd}\)

    \(\implies f^{-1}(Y\smin U)\ \TR{clsd\ by\ hypothesis}\)

    \(=X\smin f^{-1}(U)\)
    
    \(\implies f^{-1}(U)\ \TR{open}\)
\end{proof}
\ebox


\bboxprop
\begin{prop}[Continuity preserves Convergence in Metric Spaces]
    \((X,d),\ (Y,d'),\ f:X\to Y\).

    Then \(f\) is cts \TU{iff}

    \(f(x_n)\to f(x)\ \TR{whenever}\ (x_n)\ssq X,\ x_n\to x\in X.\)
\end{prop}
\ebox

\bboxproof
\begin{proof}[\((\implies)\) Proof]
    Assume \(f\) is cts.

    \(\llet:(x_n)\ssq X,\ x_n\to x\in X\).

    \(\llet:\e>0\).

    \(\TR{Consider}:U=B_\e(f(x))\).

    \(\implies x\in f^{-1}(U)\ \is\ \TR{open\ by\ cts}\).

    \(\implies\exists r>0,\ B_r(x)\ssq f^{-1}(U)\ \TR{by\ openess}\)

    \(\TR{Since}\ x_n\to x,\ \exists N\in\N\st n\ge N\implies f(x_n,x)<r\)

    \(\TR{Hence},\ n\ge N\implies x_n\in f^{-1}(U)\)

    \(\implies\iif\ n\ge N\ \tthen\ d'(f(x_n), f(x))<\e\)

    \(\implies f(x_n)\to f(x)\).
\end{proof}
\ebox


\bboxproof
\begin{proof}[\((\limplies)\) Proof]
    \[
        \TR{Assume}: f(z_n)\to f(z)\ \iif\ z_n\to z
    \]
    \[
        \llet:A\ssq X\ \TR{be\ open}
    \]
    \[
        \TR{Show}: f(\cls{A})\ssq\cls{f(A)}
    \]
    \[
        \llet: y\in f(\cls{A})\st y=f(x)
    \]
    \[
        \llet: (a_n)\ssq A\st a_n\to x
    \]
    \[
        \implies f(a_n)\to f(x)=y
    \]
    \[
        \implies y\in \cls{f(A)}
    \]
\end{proof}
\ebox

\subsection{Bounded Linear Maps}
\bboxdefn
\begin{defn}[Operator Norm, Bounded Linear Map]
    \(V,W,\ \TR{NVS},\ T:V\to W\ \is\ \TR{linear}\).

    \(T\ \is\ \TR{bd}\ \iff\)
    \[
        \|T\|_\TR{op}:=\sup\{
            \|T(x)\|:\|x\|=1
        \} < \infty
    \]
\end{defn}
\ebox

\bboxprop
\begin{prop}
    \(B(V,W):=\{T:V\to W\ |\ T\ \TR{linear\ and\ bd}\}\) is a vector space.

    Prove \(\|\cdot\|_\TR{op}\) is a norm on \(B(V,W)\).
\end{prop}
\ebox

\bboxproof
\begin{proof}
    Show
    \begin{enumerate}
        \item \(\|sT\|_\TR{op} = |s|\|T\|_\TR{op}\)
        \item \(\|T\|_\TR{op}=0\iff T=0\)
        \item \(\|T+S\|_\TR{op}\le\|T\|_\TR{op}+\|S\|_\TR{op}\)
    \end{enumerate}

    \TB{1:} \(\|sT\|_\TR{op}=\sup\{\|sT(x)\|:\|x\|=1\}\)

    \(=\sup\{|s|\|T(x)\|:\|x\|=1\}\)

    \(=|s|\|T\|_\TR{op}\)

    \TB{2 (\(\boldsymbol\implies\)):}
    \(\|T\|_\TR{op}=\sup\{\|T(x)\|:\|x\|=1\}=0\)

    \(\implies \|T(x)\|=0,\ \iif\ \|x\|=1\)

    \(\implies T(x)=0,\ \iif\ \|x\|=1\)

    \(\implies T=0_\TR{op}\)

    \TB{2 (\(\boldsymbol\limplies\)):}
    \(T=0_\TR{op}\)

    \(\implies T(x)=\vec 0\implies \|T(x)\|=0\)

    \(\implies \sup\{\|T(x)\|:\|x\|=1\} = 0\)

    \TB{3:}
    \(\|T+S\|_\TR{op}=
    \sup\{\|(T+S)(x)\|:\|x\|=1\}\)

    \(=\sup\{\|T(x) + S(x)\|:\|x\|=1\}\)

    \(\le\sup\{\|T(x)\|+\|S(x)\|:\|x\|=1\}\)
    
    \(=\|T\|_\TR{op}+\|S\|_\TR{op}\)
\end{proof}
\ebox

\bboxnote
\begin{note}
    \(T\in B(V,W).\)

    \(\iif\ \vec 0\neq x\in V\ \tthen\ 
    \left\|T\left(\frac x{\|x\|}\right)\right\|\le\|T\|_\TR{op}\)

    \(\implies \frac1{\|x\|}\|T(x)\|\le\|T\|_\TR{op}\)

    \(\implies\|T(x)\|\le\|x\|\cdot\|T\|_\TR{op}\)
\end{note}
\ebox

\bboxprop
\begin{prop}[Continuous Linear iff Bounded Linear]
    \(V,W\ \TR{NVS},\ T:V\to W\ \TR{linear}\).

    \(\tthen\ T\ \is\ \TR{cts}\iff T\ \is\ \TR{bd}\)
\end{prop}
\ebox

\bboxproof
\begin{proof}[\((\neg\limplies\neg)\) Proof]
    Assume \(T\) isn't bd. So
    \(\forall\|x_n\|=1,\ \|T(x_n)\|\ge n\).

    \(\TR{Consider}: \|\frac1nx_n\|=\frac1n\to 0\)

    \(\|T(\frac1nx_n)\|=\frac1n\|T(x_n)\|\ge\frac1nn\ge1\)

    So \(T\) doesn't preserve convergence, so \(T\) isn't cts.
\end{proof}
\ebox

\bboxproof
\begin{proof}[\((\limplies)\) Proof]
    Assume \(T\) is bd. Show \(T\) is cts.
    \[
        \|T\|_\TR{op}=\sup\{\|T(x)\|:\|x\|=1\}<\infty
    \]
    Let \((x_n)\ssq V\st x_n\to x\in V\).

    \(\|T(x_n) - T(x)\|=\|T(x_n-x)\|\le\|x_n-x\|\|T\|_\TR{op}<\frac\e{\|T\|_\TR{op}}
    \|T\|_\TR{op}=\e\)
\end{proof}
\ebox

\subsection{More Continuity}


\bboxdefn
\begin{defn}[Uniform Continuity]
    \((X,d),\ (Y,d'),\ f:X\to Y\).

    \(f\) is \TB{uniform continuous} \TU{iff}

    \(\forall\e>0\ \exists\delta>0\st
    d'(f(a),f(b))<\e\ \iif\ a,b\in X\ \TR{w/}\ 
    d(a,b)<\delta\)
\end{defn}
\ebox

\bboxnote
\begin{note}
    \(f\) is unif cts \TU{iff}

    \(\forall\e>0\ \exists\delta>0\) which works to establish
    continuity at every \(b\in X\).
\end{note}
\ebox

\bboxdefn
\begin{defn}[Lipschitz]
    \((X,d)\ (Y,d'),\ f:X\to Y.\)

    We say \(f\) is \TB{Lipschitz} \TU{iff}

    \(\exists M>0,\ d'(f(x),f(x))\le Md(x,y),\ \forall x,y\in X\).
\end{defn}
\ebox

Kind of like uniform uniform cts.

\bboxprop
\begin{prop}[Lipschitz implies Uniform Continuous]
    \(f:X\to Y\) Lipschitz then \(f\) is unif cts.
\end{prop}
\ebox

\bboxproof
\begin{proof}
    \(\llet:\e>0,\ M>0\ \TR{be\ the\ Lipschitz\ constant\ for}\ f\).

    \(\llet:\delta=\frac\e M\)

    \(\TR{Assume}:d(a,b)<\delta,\ a,b\in X.\)

    \(\implies d'(f(a),f(b))\le Md(a,b)<\e\)
\end{proof}
\ebox

\bboxexam
\begin{exam}
    \(f:[0,1]\to\R,\ f(x)=\sqrt x\).

    \(\TR{Claim}:f\ \is\ \TR{unif\ cts}\)
    \bboxnote
    \begin{note}
        \begin{align*}
            |\sqrt x - \sqrt y|^2 &= |\sqrt x - \sqrt y||\sqrt x - \sqrt y|\\
                                  &\le|\sqrt x+\sqrt y||\sqrt x-\sqrt y|\\
                                  &=|x-y|
        \end{align*}
    \end{note}
    \ebox
    \(\llet:\e>0,\ \delta=\e^2\).

    \(\iif\ a,b\in[0,1]\ \TR{w/}\ 
    |a-b|<\delta=\epsilon^2\
    \tthen\ |\sqrt a-\sqrt b|<\e\)
\end{exam}
\ebox

\bboxexam
\begin{exam}
    \(\TR{Claim}:f\ \is\ \TR{not\ Lipschitz}\)

    \(\TR{Suppose}:f\ \is\ \TR{Lipschitz}\)

    \(\TR{WLOG,\ assume}:M>1\)

    \(\frac1{M^4}\in[0,1]\)

    \(\left|\frac1{\sqrt{M^4}}-0\right|\le M\left|\frac1{M^4}-0\right|\)

    \(\implies\frac1{M^2}\le\frac1{M^3}\implies M^3\le M^2\)

    That's a contradiction.
\end{exam}
\ebox


% <++>
\end{document}


























% scrolloff buffer

%%% Local Variables:
%%% mode: LaTeX
%%% TeX-master: t
%%% End:
