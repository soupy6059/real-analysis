\documentclass[12pt]{report}

% Page Formatting
\usepackage[a4paper, margin=1in]{geometry}  % Page size and margins
\usepackage{setspace}  % Line spacing
\onehalfspacing  % 1.5 spacing
\usepackage{xcolor}
\usepackage{pagecolor}

% Encoding and Language
\usepackage[utf8]{inputenc}  % Encoding
\usepackage[english]{babel}  % Language
\usepackage{booktabs}
\usepackage{tabularx} % tables

% Math Packages
\usepackage{amsmath, amssymb, amsthm, mathrsfs, cancel}  % Advanced math support

% Hyperlinks

% Bibliography
\usepackage[numbers]{natbib}  % Citation style

% boxes
\usepackage[most]{tcolorbox} % this includes tikz i believe "t"
\usetikzlibrary{math} % variables for math

\title{Real Analysis}
\author{Carter Aitken}
\date{2025-05-05}
\pagecolor{black!20!white}

% VIM MACROES
% swap everything in brackets
% vi(F,"gyvi(F,��5vt)"hydi("hPF,xxf)i, ��5"gpx

\newtheoremstyle{theoremdd}%
    {}%
    {}%
    {\it}%
    {}%
    {\bf}%
    {}%
    { }%
    {\thmname{#1}\thmnumber{ #2}%
    \thmnote{: #3\addcontentsline{toc}{subsubsection}{{\it#1}: #3}}. }
\theoremstyle{theoremdd}

% PICTURE EXAMPLE
%\tikzmath{
%    \CAwidth = 10;
%    \CAarrowLen = 2;
%    \CAoriginX = 2;
%    \CAoriginY = 0;
%    \CAmargin = 0.5;
%    \CAhorAdj = 2;
%}
%\begin{tikzpicture}
%    \draw[thick] (\CAoriginX,\CAoriginY) -- (\CAwidth+\CAoriginX,\CAoriginY); % board in river
%    \fill[black] (\CAoriginX + \CAwidth / 2,\CAoriginY) circle (0.1cm); % centre dot
%%    \draw[->] (0,0) -- (\CAarrowLen,\CAarrowLen) node[anchor=north east] {\((p-\e)^2\)}; % left arrow
%    \draw (-\CAarrowLen+\CAoriginX,-\CAarrowLen+\CAoriginY) node (leftArrow) {\((p-\e)^2\)};
%    \draw[->] (\CAoriginX,\CAoriginY) -- (leftArrow);
%    \draw (\CAarrowLen+\CAoriginX+\CAwidth,\CAoriginY+\CAarrowLen) node (rightArrow) {\((p+\e)^2\)};
%    \draw[->] (\CAoriginX+\CAwidth,\CAoriginY) -- (rightArrow);
%    \draw (\CAoriginX + \CAwidth / 2,\CAoriginY-\CAmargin) node {\((p,q)\)};
%    \draw (\CAoriginX, \CAoriginY+\CAmargin) node {\((p-\e,q)\)};
%    \draw (\CAoriginX + \CAwidth,\CAoriginY-\CAmargin) node {\((p+\e,q)\)};
%    \draw (\CAoriginX + \CAwidth / 2 + \CAmargin * 2, \CAoriginY) node (A) {};
%    \draw (\CAoriginX + \CAwidth / 2 - \CAmargin * 2, \CAoriginY) node (B) {};
%    \draw[thick,->] (A) to [bend right=45] (B);
%\end{tikzpicture}

% thm type declarations
\newtheorem{thm}{Theorem}[section]
\newtheorem{lem}{Lemma}[section]
\newtheorem{defn}{Definition}[section]
\newtheorem{coro}{Corollary}[section]
\newtheorem{prop}{Proposition}[section]

\definecolor{berkblue}{rgb}{0.02,0.18,0.37}
\definecolor{indigo}{rgb}{0,0.33,0.47}
\definecolor{jungle}{rgb}{0.02,0.66,0.49}

\definecolor{darkgoldenrod}{rgb}{0.72, 0.53, 0.04}
\newcommand{\bbox}{\begin{tcolorbox}[colback=blue!20!white,colframe=blue]}
\newcommand{\bboxproof}{\begin{tcolorbox}[colback=darkgoldenrod!20!white,colframe=darkgoldenrod]}
\newcommand{\bboxex}{\begin{tcolorbox}[colback=red!20!white,colframe=red]}
\newcommand{\bboxnote}{\begin{tcolorbox}[colback=yellow!20!white,colframe=yellow]}
\newcommand{\ebox}{\end{tcolorbox}}

% document tools


\NewDocumentCommand{\sff}{}{\mathrm{I\!I}}
\newcommand{\R}{\mathbb R}
\newcommand{\N}{\mathbb N}
\newcommand{\Q}{\mathbb Q}
\newcommand{\Z}{\mathbb Z}

\newcommand{\mat}[2][bmatrix]{
    \begin{#1} #2 \end{#1}
}

\newcommand{\inn}[1]{\left\langle #1\right\rangle}

\newcommand{\T}[1]{\text{#1}}

\newcommand{\TI}[1]{\textit{#1}}

\newcommand{\TB}[1]{\textbf{#1}}

\newcommand{\TU}[1]{\text{\underline{#1}}}

\newcommand{\TM}[1]{(\T{#1})}

\newcommand{\st}{\T{ s/t }}
\newcommand{\ot}{\leftarrow}
\newcommand{\limplies}{\Longleftarrow}
\newcommand{\is}{\ot}

\newcommand{\mycornmat}[5][bmatrix]{
    \begin{#1}
        #2 & \cdots & #3 \\
        \vdots & \ddots & \vdots \\
        #4 & \cdots & #5
    \end{#1}
}

\newcommand{\subeq}{\subseteq}
\newcommand{\supeq}{\supseteq}

\newcommand{\CCOS}{connected compact orientable surface}
\newcommand{\cts}{\mathrm{cnts}}
\newcommand{\nbhd}{\mathrm{nbhd}}
\newcommand{\open}{\mathrm{open}}
\newcommand{\spc}{\T{ }}
\newcommand{\md}{\,\mathrm{d}}
\newcommand{\e}{\epsilon}
\newcommand{\de}{\delta}
\newcommand{\al}{\alpha}
\newcommand{\be}{\beta}
\newcommand{\om}{\Omega}
\newcommand{\PP}{\mathcal P}
\newcommand{\XX}{\mathcal X}
\newcommand{\QQ}{\mathcal Q}


% Document Starts
\begin{document}

\maketitle 

\begin{abstract}
  Real Analysis the study of approximation on the reals.
\end{abstract}

\tableofcontents

\pagebreak
% <++>

\chapter{Cardinality}
\section{Brief Motivation}
We want to build a metric space to measure the distance between objects.

We need
\begin{enumerate}
  \item set $X$ of objects.
  \item need to measure closeness. func \(d:X\times X\to[0,\infty)\) s/t
    \begin{enumerate}
      \item \(d(x,y)=0\iff x=y\)
      \item \(d(x,y)=d(y,x)\)
      \item \(d(x,z)\le d(x,y)+d(y,z)\)
    \end{enumerate}
\end{enumerate}

We call \(d\) a metric on \(X\). \((X,d)\) is a \TB{metric space}.
\bboxex
\((\R,+'\circ^\circ(\sqrt{\,}^=)-)\) is a metric space. Note the BQN notation.
\ebox

\section{Function Theory}
\bbox
\begin{defn}[Injection]
  Let \(A,B\) be non-empty sets. We say \(f:A\to B\) is injective \TB{iff}
  \(\forall a,b\in A\quad f(a)=f(b)\implies a=b\)
\end{defn}
\ebox

\bbox
\begin{defn}[Surjection]
  \(f:A\to B\) is a surjection if \(\forall b\in B\quad\exists a\in A\st f(a)
  = b\).
\end{defn}
\ebox

\bbox
\begin{defn}[Bijective]
  \(f:A\to B\) is bijective \TB{iff} its injective and surjective.
\end{defn}
\ebox


\bbox
\begin{defn}[Invertable]
  \(f:A\to B\) is invertable \TB{iff} \(\exists g:B\to A\st g(f(a))=a\T{ and }
  f(g(b))=b\quad\forall a\in A,\,b\in B\).
\end{defn}
\ebox
We write \(g=f^{-1}\) and call it "the" inverse.


\bbox
\begin{prop}
  \(f:A\to B\) is invertable \TB{iff} f is bijective.
\end{prop}
\ebox

\begin{proof}
  \((\implies)\) \(f\) is invertable. Suppose \(f(a)=f(b)\). We'll show \(a=b\).
  \[f^{-1}f(a))=f^{-1}f(b))\]
  \[\implies a=b\]
  
  Now we'll show \(\forall b\in B\;\exists a\in A\;f(a)=b\).
  \[a=f^{-1}(b)\implies\T{there is way to get from b to a, and it's }f^{-1}\]

  \((\limplies)\) Assume \(f\is(\T{bijective})\). We'll construct \(f\)'s
  inverse. For \(b\in B\) let \(a_b\) be the unique element of \(A\st f(a_b)=b\).
  \(a_b\) exists b/c of surjectivity of f, and it's unique b/c of injectivity.
  \[g:=\{g:A\to B,\,g(b)=a_b\}\]
  \[f(g(b))=f(a_b)=b\]
  \[g(f(a_b))=g(b)=a_b\]
  \[\implies g=f^{-1}\]
\end{proof}

\bbox
\begin{prop}
  \(\exists(\T{injection})\,f:A\to B\iff\exists(\T{surjection})\,g:B\to A\)
\end{prop}
\ebox

\begin{proof}
  \((\implies)\) Suppose \(f:A\to B\is(\T{injective})\). Let \(b\in B\).
  
  Case 1: \(b\in f(A)\). 

  Let \(g(b)\) be the unique element of \(A\st f(g(b))=b\), unique b/c 
  \(f\is\TM{injective}\)

  Case 2: \(b\not\in f(A)\).

  Fix any \(z\in A\). Let \(g(b)=z\).
  
  \begin{equation}
    \implies g(b)=\begin{cases}f^{=}(b)&b\in f(A)\\ z&b\not\in f(A)\end{cases}
  \end{equation}

  We claim \(g\) is a surjection. 
  So we have to show
  \(\forall a\in A,\,\exists b\in B\st g(b)=a\)
  Let \(a\in A\st f(a)\in B\).
  \[g(f(a))\implies f(g(f(a)))=f(a)\]
  \[\TM{injective}\implies g(f(a))=a\]
  \[\implies g\is\TM{surjective}\]

  \(\limplies\) Suppose \((g:B\to A)\is\TM{surjective}\).
  \(\forall a\in A\T{ choose }b_a\in B\st g(b_a)=a\).
  \(f:=\{f:A\to B\quad f(a)=b_a\}\). Suppose
  \[f(x)=f(y)\]
  \[\implies b_x=b_y\]
  \[\implies g(b_x)=g(b_y)\]
  \[\implies x=y\]
  \[\implies f\is\TM{injective}\]
\end{proof}

% <++>
\end{document}


























% scrolloff buffer
